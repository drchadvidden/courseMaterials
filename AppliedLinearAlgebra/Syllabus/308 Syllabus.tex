\documentclass [11pt]{article}

\usepackage{hyperref}

\setlength{\evensidemargin}{0.0cm}
\setlength{\oddsidemargin}{0.0cm}
\setlength{\topmargin}{-1.75cm}
%\setlength{\baselineskip}{20pt}
\setlength{\textwidth}{17cm}
\setlength{\textheight}{24.5cm}
\hoffset = -0.75cm
\pagestyle{empty}


\begin {document}
\begin {center}
\Large \bfseries MTH 308
Linear Algebra and Differential Equations \normalfont \normalsize \\
Spring 2023 \\
{\bf Lecture} MTW in Centennial 2311, {\bf Lab} Th in Wing 6
\end {center}
\ \\
\noindent
%%%%%%%%%%%%%%%%%%%%%%%%%%%%%%%%%%%%%%
\bfseries Instructor: \normalfont Chad Vidden \\
\bfseries Office: \normalfont 1009 Cowley Hall \\
\bfseries Email: \normalfont cvidden@uwlax.edu \\
\bfseries Office Hours: \normalfont See Canvas for schedule, or by appointment, virtual office hours by request \normalfont  \\
\bfseries Course Website: \normalfont UWL Canvas Site: http://www.uwlax.edu/canvas/\\
\bfseries Textbook: \normalfont \itshape Linear Algebra and its Applications by Lay et al, 5th Edition \normalfont \\
\ \\
%%%%%%%%%%%%%%%%%%%%%%%%%%%%%%%%%%%%%%
\bfseries Course Description: \normalfont This course will study linear algebra with emphasis on computer programming and applications. Specific topics include systems of linear equations, matrix operations, linear independence, linear transformations, matrix factorization, vector spaces and subspaces in $R^n$, basis and dimension, determinants, eigenvalues and eigenvectors, diagonalization, systems of first order linear differential equations, dynamical systems, inner products and orthogonality, least squares, and singular value decomposition. Software will be integrated throughout the course to complement mathematical content. Lect. 3 credit, Lab. 1 credit. \\ \ \\
%%%%%%%%%%%%%%%%%%%%%%%%%%%%%%%%%%%%%%
\bfseries Prerequisite: \normalfont
Grade of "C" or better in MTH 208 or MTH 265 or (MTH 207 and CS 225 or MTH 225). CS 120 or concurrent enrollment highly recommended. \\
\ \\
%%%%%%%%%%%%%%%%%%%%%%%%%%%%%%%%%%%%%%
\bfseries Course Contents: \normalfont Linear equations in linear algebra (Chapter 1);  Matrix algebra (Chapter 2);  Determinants (Chapter 3);  Eigenvalues and eigenvectors (Chapter 5); Orthogonality and least squares (Chapter 6);  Symmetric matrices and quadratic forms (Chapter 7). \\
\ \\
%%%%%%%%%%%%%%%%%%%%%%%%%%%%%%%%%%%%%%
\bfseries Course Objectives: \normalfont By the end of this course, the student should be able to:
\begin {itemize}
\setlength{\itemsep}{1pt}
\setlength{\parskip}{0pt}
\setlength{\parsep}{0pt}
\item Be computational proficient using both hand calculations as well as computer programming to solve problems of linear algebra.
\item Demonstrate understanding of the theory of linear algebra.
\item Solve problems which apply linear algebra and first order systems of differential equations to fields such as physics, engineering, chemistry, biology, economics, and computer science.
\end{itemize}
\ \\
%%%%%%%%%%%%%%%%%%%%%%%%%%%%%%%%%%%%%%
\bfseries Learning Outcomes: \normalfont By the end of this course, the student should be able to:
\begin {itemize}
\setlength{\itemsep}{1pt}
\setlength{\parskip}{0pt}
\setlength{\parsep}{0pt}
\item Demonstrate proficiency of solving systems of linear equations by Gaussian elimination. 
\item Demonstrate proficiency of matrix calculations including multiplication, linear transformations, matrix inverses, and determinants.
\item Demonstrate proficiency in calculation of eigenvalues and eigenvectors as well as matrix diagonalization.
\item Demonstrate proficiency in solving linear systems of differential equations.
\item Demonstrate understanding and applying the singular value decomposition of a matrix.
\item Demonstrate understanding of vector spaces and subspaces of $R^n$ in connection to the row, column, and null space of a matrix.
\item Demonstrate understanding of linear independence basis, dimension and rank when applied to vector spaces of $R^n$.
\item Demonstrate understanding of inner product, norm, and orthogonality.
Write computer scripts to solve practical problems and illustrate theoretical results.
\item Synthesize computational, programming, and theoretical results with applied projects.
\end{itemize}
\pagebreak

\noindent
%%%%%%%%%%%%%%%%%%%%%%%%%%%%%%%%%%%%%%
\bfseries Grading Policy: \normalfont
Each portion of your grade is worth the following out of a {\bf possible 800 points}. \vspace{-2mm}
\begin {itemize}
\setlength{\itemsep}{1pt}
\setlength{\parskip}{0pt}
\setlength{\parsep}{0pt}
\item Quizzes: (12 points) x (10 quizzes) = 120 points
\item Lab Assignments: (15 points) x (10 labs) = 150 points 
\item Projects: (40 points) x (2 projects) = 80 points
\item Chapter Exams: (100 points) x (3 exams) = 300 points
\item Comprehensive Final Exam: 150 points
\end {itemize} \vspace{-2mm}
Your final letter grade will be no worse than as listed below. There will be {\bf no curve at any point.} 
\begin{center}
\begin{tabular}{ccc}
A: & 736 - 800 points & (92 \% - 100 \%)\\
A/B: & 704 - 736 points & (88 \% - 92 \%)\\
B: & 656 - 704 points & (82 \% - 88 \%)\\
B/C: & 624 - 656 points & (78 \% - 82 \%)\\
C: & 560 - 624 points & (70 \% - 78 \%)\\
D: & 480 - 560 points & (60 \% - 70 \%)\\
F: & 0 - 480 points & (0 \% - 60 \%)
\end{tabular}
\end{center}
\ \\
%%%%%%%%%%%%%%%%%%%%%%%%%%%%%%%%%%%%%%
\bfseries Textbook Homework: \normalfont Homework from each section will be assigned from the textbook. Problems will be used exactly for quiz questions, so completing all homework prior to taking a quiz is recommended. The complete collection all prior homework problems will be accepted on Exam days for bonus. Students are encouraged to work in groups and ask questions in class or office hours.
\\
\ \\
%%%%%%%%%%%%%%%%%%%%%%%%%%%%%%%%%%%%%%
\bfseries Self-Graded Quizzes: \normalfont Self-graded quizzes will be assigned each Monday and due Wednesdays. See Canvas for self-graded quiz guidelines. \\
\ \\
%%%%%%%%%%%%%%%%%%%%%%%%%%%%%%%%%%%%%%
\bfseries Lab Assignments: \normalfont Thursday of each week is lab day. Lab assignments will be completed using the R programming language and Jupyter notebooks. Labs are assigned at the beginning of class on Thursdays and due the following Monday at midnight. Students are encouraged to work together on labs. \emph{No prior coding experience is required.} See the course Canvas website for details. \normalfont \\
\ \\
%%%%%%%%%%%%%%%%%%%%%%%%%%%%%%%%%%%%%%
\bfseries Projects: \normalfont  Two group projects will be assigned throughout the semester and will focus on modern applications of concepts from this class. See Canvas webiste for details.  \normalfont \\
\ \\
%%%%%%%%%%%%%%%%%%%%%%%%%%%%%%%%%%%%%%
\bfseries Exams: \normalfont Three in-class exams will be given covering content grouped by textbook chapters. Exam problems and structure will closely resemble that of in-class quizzes. Makeup exams will not be given without acceptable reason. \\
\ \\
%%%%%%%%%%%%%%%%%%%%%%%%%%%%%%%%%%%%%%
\bfseries Final Exam: \normalfont The final comprehensive exam will be on {\bf Wednesday, May 11 from 4:45-6:45pm}.  \\
\ \\
%%%%%%%%%%%%%%%%%%%%%%%%%%%%%%%%%%%%%%
\bfseries Important Semester Dates: \normalfont UWL academic calendar \\ \url{https://www.uwlax.edu/records/dates-and-deadlines/} \\
\ \\
%%%%%%%%%%%%%%%%%%%%%%%%%%%%%%%%%%%%%%
\bfseries Resources: \normalfont 
\begin{itemize}
\setlength{\itemsep}{1pt}
\setlength{\parskip}{0pt}
\setlength{\parsep}{0pt}
\setlength{\itemsep}{1pt}
\setlength{\parskip}{0pt}
\setlength{\parsep}{0pt}
\item Desmos graphing tool: \url{https://www.desmos.com/}
\item Geogebra graphing tool: \url{https://www.geogebra.org/}
\item MIT linear algebra: \url{https://ocw.mit.edu/courses/mathematics/18-06-linear-algebra-spring-2010/}
\item MIT applied matrix methods: \url{https://ocw.mit.edu/courses/mathematics/18-065-matrix-methods-in-data-analysis-signal-processing-and-machine-learning-spring-2018/}
\item Computational tool: \url{http://www.wolframalpha.com/}
\item Textbook homework answers: \url{https://www.slader.com/}
\item AI generated solutions: \url{https://www.symbolab.com/}
\end{itemize}
%%%%%%%%%%%%%%%%%%%%%%%%%%%%%%%%%%%%%%
\bfseries Wisdom: \normalfont Heed the following advice.  \begin{itemize}
\setlength{\itemsep}{1pt}
\setlength{\parskip}{0pt}
\setlength{\parsep}{0pt}
%%%%%%%%%%%%%%%%%%%%%%%%%%%%%%%%%%%%%%
\item How should I use the textbook?  \vspace{-1mm}
\begin {itemize}
\setlength{\itemsep}{1pt}
\setlength{\parskip}{0pt}
\setlength{\parsep}{0pt}
\item Browse a section before seeing it in lecture, again after. Work through all text examples in detail before attempting homework problems (don't copy, write down problem, try to solve, check your work against text solution). Consider all homework problems. Carefully work through problems which you are unsure of and keep detailed solutions in a dedicated notebook.
\end{itemize}
%%%%%%%%%%%%%%%%%%%%%%%%%%%%%%%%%%%%%%
\item What if I get stuck on homework?  \vspace{-1mm}
\begin {itemize}
\setlength{\itemsep}{1pt}
\setlength{\parskip}{0pt}
\setlength{\parsep}{0pt}
\item Complete in the following order: Re-read notes and textbook section. Move on to the next problem and revisit. Step away for 30 mins and try again. Bring question to your weekly HW group meeting. Visit the Mathematics Learning Center in 273 Murphy Library (\url{https://www.uwlax.edu/murphy-learning-center/}). Come to office hours.
\end{itemize}
%%%%%%%%%%%%%%%%%%%%%%%%%%%%%%%%%%%%%%
\item What should I do during class?  \vspace{-1mm}
\begin {itemize}
\setlength{\itemsep}{1pt}
\setlength{\parskip}{0pt}
\setlength{\parsep}{0pt}
\item Review yesterday's notes when sit down. Consider the warm up problem. Take new notes, but make them your own. Add spoken comments, your own thoughts and reminders, other student questions, etc.
\item Think. Question in your mind what is said. What do/don't you understand? Think of questions to ask. Talk to neighbor during working sessions. Ask questions.
\end{itemize}
%%%%%%%%%%%%%%%%%%%%%%%%%%%%%%%%%%%%%%
\item What should I expect from this class? \vspace{-1mm}
\begin{itemize}
\setlength{\itemsep}{1pt}
\setlength{\parskip}{0pt}
\setlength{\parsep}{0pt}
\item Fast pace and heavy workload. Self study, freedom to learn in your own style, and interactive lecture. Less feedback, grading, and interaction. No in - class review for exams. Longer, harder exams and lower course grades. No calculators or formula sheets.
\end{itemize}
%%%%%%%%%%%%%%%%%%%%%%%%%%%%%%%%%%%%%%
\item Why do I need math? \\ 
Focus on growing these skills throughout the semester.
\begin {itemize}
\setlength{\itemsep}{1pt}
\setlength{\parskip}{0pt}
\setlength{\parsep}{0pt}
\item Attention to detail / accuracy of result. Can you consistently perform long calculations without making any mistakes?
\item Communication / writing style. Does your writing clearly and concisely explain what you know? When writing solutions, is it clear to the instructor you understand content?
\item Logical thinking / ability to abstract. Does seeing a concept applied to only a few examples allow you to execute it in a number of analogous settings?
\end{itemize}
\end{itemize}
%%%%%%%%%%%%%%%%%%%%%%%%%%%%%%%%%%%%%%
\bfseries UWL syllabus polocies \normalfont \\ 
See \url{https://www.uwlax.edu/info/syllabus/} for important statements. Especially not the comments related to Covid-19, academic dishonesty, and student accommodations. \\ \ \\
%%%%%%%%%%%%%%%%%%%%%%%%%%%%%%%%%%%%%%
\bfseries Disclaimer: \normalfont This syllabus is subject to change at any time. Changes will be announced, and an updated syllabus will be posted on the course website. \\ \ \\
%%%%%%%%%%%%%%%%%%%%%%%%%%%%%%%%%%%%%%
%%%%%%%%%%%%%%%%%%%%%%%%%%%%%%%%%%%%%%
\bfseries Covid Pandemic Consideration \normalfont \\ 
I recognize this is a difficult time for every student in this course, and some students may be operating under additional constraints compared to normal semesters. As an instructor, I will be understanding and flexible of individual student circumstances regarding course deadlines. Do reach out to me if you are struggling to keep up. I will do everything I can to help you. 
%%%%%%%%%%%%%%%%%%%%%%%%%%%%%%%%%%%%%%
%\section*{OPTIONAL Textbook homework list}
%{\large \begin{center}
%\begin{tabular}{|l|l|}
%\hline
%{\small {\bf Section}}		& ODD PROBLEMS ONLY UNLESS INDICATED \\ \hline
%Section 2.1 & 1, 3, 5, 9 \\ \hline
%Section 2.2 & 1-11, 16, 17, 19, 31, 33, 39, 42, 50, 53, 55* \\ \hline
%\end{tabular}
%\end{center}}



\end {document} 