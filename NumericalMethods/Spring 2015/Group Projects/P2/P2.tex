\documentclass[addpoints, 11pt]{exam}

\usepackage{amsmath}
\usepackage{amssymb}
\usepackage{graphicx}
\usepackage{hyperref}
\usepackage{fancyhdr}

\pagestyle{fancy}

\rhead{{\bf Assigned:} Wednesday, March 25, 2015 \\{\bf Due:} Week of April 20, 2015}

\printanswers
%\noprintanswers
\newcommand{\ds}{\displaystyle}
\newcommand{\lm}{\lim\limits}
\newtheorem{Definition}{Definition}

\begin{document}
\vspace{100mm}
\begin{center} \Large
MTH 371: Group Project 2 \\ Least Squares Approximation and Sports Ranking \normalsize
\end{center}
\ \\
\noindent GENERAL GROUP PROJECT GUIDELINES: 
\begin{itemize}
\item Group project assignments should be a collaborative effort. All should participate in discussion and solution writing. \vspace{-2mm}
\item Two weeks after the project is assigned, your group will meet with Dr. Vidden to discuss. All members must be present. Your grade will be determined at the end of the meeting. \vspace{-2mm}
\item Each student should keep group project solutions in a dedicated notebook. Bring this notebook to your weekly meeting to discuss your findings. For coded solutions, bring a laptop to your weekly meeting. Have the laptop ready before the start of the meeting. \vspace{-2mm}
%\item Clearly label all plots (title, $x$-axis, $y$-axis, legend). Use the \verb1subplot 1command when comparing 2 or more plots.
\end{itemize}
\ \\

With this project, we consider a least squares approach for ranking sports teams. This method known as the Massey method and was developed as an undergraduate thesis by Kenneth Massey in 1997. Since then, this method has progressed and is now used by the Bowl Championship Series system to select NCAA football bowl matchups.

\begin{questions}

%%%%%%%%%%%%%%%%%%%%%%%%%%%%%%%%%%%%%%%%%%%%%%%%%%%%%%%%%%%%%%%%%%%%%
\question In this class, we have solved nonsingular systems $A\vec{x}=\vec{b}$ with $n$ equations and $n$ unknowns. If on the other hand there are \emph{more equations than unknowns}, the system probably has no solution. Instead, we might wish to find an approximate solution $A\vec{x}\approx \vec{b}$. The \emph{least squares problem} is to minimize the residual in the $\ell^2$ norm, $\ds \|\vec{b}-A\vec{x}\|_2$. Recall, $\|\cdot\|_2$ denotes the Euclidean norm. For this problem, we solve the least squares problem using calculus. Assume dimensions $\ds A_{m\times n}, \vec{x}_n,$ and $\vec{b}_m$.
\begin{parts}
\part Why is this problem called the \emph{least squares problem}?
\part To minimize $\ds \|\vec{b}-A\vec{x}\|_2$, differentiate in each component $x_k$ and set equal to zero. Why must there only be one minimum here?
\part From (b), show that for each $k=1,\dots,n$,
\[
\sum_{i=1}^m a_{ik}\left( \sum_{j=1}^na_{ij}x_j \right) = \sum_{i=1}^m a_{ik}b_i.
\]
\part From (c), show that $A^TA\vec{x}=A^T\vec{b}$. This systems is called the \emph{normal equations}.
\end{parts}

\pagebreak

%%%%%%%%%%%%%%%%%%%%%%%%%%%%%%%%%%%%%%%%%%%%%%%%%%%%%%%%%%%%%%%%%%%%%
\question Consider the following set of basketball matchups for teams $T_1, \dots, T_5$. \vspace{-6mm}
\begin{center}
\begin{tabular}{|c|c|c|c|c|c|c|c|}
\hline
\	& Duke($T_1$) & Miami($T_2$) & UNC($T_3$) & UVA($T_4$) & VT($T_5$) & Record & Point Differential \\
\hline
Duke($T_1$) &	& 7-52 & 21-24 & 7-38 & 0-45 & 0-4 & -124\\
\hline
Miami($T_2$) & 52-7 &	& 34-16 & 25-17 & 27-7 & 4-0 & 91\\
\hline
UNC($T_3$)  & 24-21 & 16-34 &	& 7-5 & 3-30 & 2-2 & -40\\
\hline
UVA($T_4$)   & 38-7 & 17-25 & 5-7 &	& 14-52 & 1-3 & -17\\
\hline
VT($T_5$)     & 45-0 & 7-27 & 30-3 & 52-14 &	& 3-1 & 90\\
\hline
\end{tabular}
\end{center}
\begin{parts}
\part For each game, we have the equation $r_i-r_j=y_k$ where $y_k$ is the margin of victory for game $k$, and $r_i$ and $r_j$ are the ratings of teams $i$ and $j$ respectively. In other words, the difference in ratings of the two teams ideally predicts the margin of victory of a match between the teams. Construct the system $A\vec{r}=\vec{y}$ for the above matchups and show it has no solution.
\part Construct the normal equations $M\vec{r} = \vec{p}$ which solve the least squares problem. What is $M$? $\vec{p}$?
\part Matrix $M$ of part (b) is called the Massey matrix. $M$ turns out to be rather simple. In fact, it needs not be computed. It can be formed by assigning diagonal entries $M_{ii}$ as the total number of games played by team $i$. For off diagonal entries $M_{ij}, i\neq j$, these are the negation of the matchups between teams $i$ and $j$. Explain how this happens.
\part Also, vector $\vec{p}$ is rather nice as well. What happens here?
\part Unfortunately, because of the structure of matrix $M$, system $M\vec{r} = \vec{p}$ has no unique solution. Show $M$ is singular (hint: what is the row sum of $M$?). Show if $\vec{r}$ solves the system, so does $\vec{r}+\vec{c}$ for any constant vector $\vec{c}$, so there are in fact infinite solutions.
\part Massey's workaround to this problem is to add another constraint to the system stating that all ratings $r_i$ must sum to 0. Replace the last equation in your least squares system with this requirement. Solve this final system. Make sure you construct $M$ in Scilab in a \emph{general way} using the nice structure of $M$ as seen above. Your code should be easily adaptable to adding more games. Your final results should match the following table.
\begin{center}
\begin{tabular}{ccc}
\hline
Team & $r_i$ & rank\\
\hline
Duke & -24.8 & 5th\\
Miami & 18.2 & 1st\\
UNC  & -8.0 & 4th\\
UVA   & -3.4 & 3rd\\
VT     & 18.0 & 2nd\\
\hline
\end{tabular}
\end{center}
\end{parts}



%%%%%%%%%%%%%%%%%%%%%%%%%%%%%%%%%%%%%%%%%%%%%%%%%%%%%%%%%%%%%%%%%%%%%
\question BONUS extra credit problems. 
\begin{parts}
\part Visit the website http://www.masseyratings.com/ and download date from your favorite sport. Rank the teams using Massey's method.
\part Read the Colley Method paper on D2L and implement this method for the above basketball matchups.
\end{parts}

\end{questions}
\end{document} 