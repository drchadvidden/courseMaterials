\documentclass[addpoints, 11pt]{exam}

\usepackage{amsmath}
\usepackage{amssymb}
\usepackage{graphicx}
\usepackage{fancyhdr}

\pagestyle{fancy}

\rhead{{\bf Assigned:} Wednesday, April 16, 2014 \\{\bf Due:} Week of April 21, 2014}

\printanswers
%\noprintanswers
\newcommand{\ds}{\displaystyle}
\newcommand{\lm}{\lim\limits}
\newtheorem{Definition}{Definition}

\begin{document}
\vspace{100mm}
\begin{center} \Large
MTH 371: Group Project 3 \\ Gaussian Quadrature Rules \normalsize
\end{center}
\ \\
\noindent GENERAL GROUP PROJECT GUIDELINES: 
\begin{itemize}
\item Group project assignments should be a collaborative effort. All should participate in discussion and solution writing. \vspace{-2mm}
\item Each week, your group must meet with Dr. Vidden to discuss your findings. All members must be present. Your grade will be assigned at the end of the meeting. \vspace{-2mm}
\item Each student should keep group project solutions in a dedicated notebook. Bring this notebook to your weekly meeting to discuss your findings. For coded solutions, bring a laptop to your weekly meeting. Have the laptop ready before the start of the meeting. \vspace{-2mm}
%\item Clearly label all plots (title, $x$-axis, $y$-axis, legend). Use the \verb1subplot 1command when comparing 2 or more plots.
\end{itemize}
\ \\

\begin{questions}

%%%%%%%%%%%%%%%%%%%%%%%%%%%%%%%%%%%%%%%%%%%%%%%%%%%%%%%%%%%%%%%%%%%%%
\question Below is a table of the zeros of the $k$th degree Legendre polynomial for $k=2,3,4,5$. These zeros give the nodes $x_i$ for Gaussian quadrature rules on interval $[-1,1]$. Using the method of undetermined coefficients, find the corresponding $A_i$ values for each $k$ via Scilab. Summarize your results in a table.
\begin{center}
\begin{tabular}{ c||c }
$k$ & $x_i$ \\ \hline  \hline   
2 & $\pm \sqrt{\frac{1}{3}}$ \\ \hline  
3 & 0, $\pm \sqrt{\frac{3}{5}}$ \\ \hline  
4 & $\pm \sqrt{\frac{1}{7}(3-4\sqrt{0.3})}$, \quad $\pm \sqrt{\frac{1}{7}(3+4\sqrt{0.3})}$  \\ \hline  
5 &  0, \quad $\pm \sqrt{\frac{1}{9}\left(5-2\sqrt{\frac{10}{7}}\right)}$, \quad $\pm \sqrt{\frac{1}{9}\left(5+2\sqrt{\frac{10}{7}}\right)}$ 
\end{tabular}
\end{center}


%%%%%%%%%%%%%%%%%%%%%%%%%%%%%%%%%%%%%%%%%%%%%%%%%%%%%%%%%%%%%%%%%%%%%
\question With the transformation $\ds t=\frac{2x-(a+b)}{b-a}$, a Gaussian quadrature rule of the form 
$$
\int_{-1}^1 f(t)~dt \approx \sum_{i=0}^n A_i f(t_i)
$$
can be used over the interval $[a,b]$. Write a function \verb1y = GaussianQuad(f,a,b,n)1 which computes the the $n$th Gaussian quadrature rule
$$
\int_{a}^b f(x)~dx \approx \sum_{i=0}^n A_i f(x_i).
$$
Use the $x_i, A_i$ values found from problem 1 stored as a table as well as the above transformation. Assume input \verb1f1 is a defined Scilab function.

\pagebreak 

%%%%%%%%%%%%%%%%%%%%%%%%%%%%%%%%%%%%%%%%%%%%%%%%%%%%%%%%%%%%%%%%%%%%%
\question Use your code from problem 2 to compute the Gaussian quadrature approximations of the following functions for $n=1,2,3,4$.
\begin{parts}
\part $\int_0^1 \frac{1}{\sqrt{x}}~dx$
\part $\int_0^1 \frac{\sin(x)}{x} ~dx$
\end{parts}

%%%%%%%%%%%%%%%%%%%%%%%%%%%%%%%%%%%%%%%%%%%%%%%%%%%%%%%%%%%%%%%%%%%%%
\question Modify the function from problem 2 to create a composite Gaussian quadrature function \verb1y = CompGaussianQuad(f,a,b,n,m)1. This function evaluates $\ds \int_a^b f(x)~dx$ by first dividing interval $[a,b]$ into $m$ equally spaced subintervals, then applying the $n$th Gaussian quadrature on each subinterval.

%%%%%%%%%%%%%%%%%%%%%%%%%%%%%%%%%%%%%%%%%%%%%%%%%%%%%%%%%%%%%%%%%%%%%
\question Use the function from problem 4 to compute the composite Gaussian quadrature approximations of the following functions.
\begin{parts}
\part $\int_0^1 x^5~dx$ using $n=2$, $m=1,2, 10$.
\part $\int_0^1 \frac{\sin(x)}{x} ~dx$ using $n=3$,  $m=1,2,3,4$.
\end{parts}

\end{questions}
\end{document} 