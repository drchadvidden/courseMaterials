\documentclass[addpoints, 11pt]{exam}

\usepackage{amsmath}
\usepackage{amssymb}
\usepackage{graphicx}

\oddsidemargin=0in
\evensidemargin=0in
\textwidth=6.3in
\topmargin=-0.5in
\textheight=9in
\parindent=0in
\pagestyle{empty}
\thispagestyle{empty}

\usepackage{amsmath}
\usepackage{graphicx}

\printanswers
%\noprintanswers
\newcommand{\ds}{\displaystyle}
\newcommand{\lm}{\lim\limits}
\newtheorem{Definition}{Definition}

\begin{document}

Math 371 \\
Quiz 7, \ \ April 10, 2015
\hspace{2.in}
{Name:} {\underline {\hspace{2.15in}}} \\ \normalsize
\begin{questions}

\question Consider the table of function values below.
\begin{center}
\begin{tabular}{ c||c|c|c }
$x$ & 1 & 3/2 & 0 \\ \hline
$f(x)$ & 3 & 13/4 & 3
\end{tabular}
\end{center}
\begin{parts}
\part Fill out the Newton divided difference table provided below for the above function values.
\begin{center}
\begin{tabular}{ l|p{2.5cm}|p{3cm}|p{4cm} }
$x$ & $f[~\cdot~]$ & $f[~\cdot~,~\cdot~]$ & $f[~\cdot~,~\cdot~,~\cdot~]$   \\ \hline \hline 
& & & \\ 
1 & $\ds f[1]=\quad$ & $\ds f[1,3/2]=\quad$ & $\ds f[1,3/2,0]=\quad$ \\
& & & \\ 
3/2 & $\ds f[3/2]=\quad$ & $\ds f[3/2, 0]=\quad$ & \\
& & & \\ 
0 & $\ds f[0]=$ &~ &
& & & 
\end{tabular}
\end{center} \ \\
Scratch space: \\
\vfill
\part Write down, but do not simplify, the polynomial interpolant $P$ in \emph{Newton form} of $f$ through points $(1,3),~ (3/2, 13/4),~ (0,3)$. \vfill
\part Write down, but do not simplify, the polynomial interpolant $P$ in \emph{Lagrange form} of $f$ through points $(1,3),~ (3/2, 13/4),~ (0,3)$.\vfill
\end{parts}
\end{questions}

\end{document} 