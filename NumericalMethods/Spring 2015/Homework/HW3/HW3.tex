\documentclass[addpoints, 11pt]{exam}

\usepackage{amsmath}
\usepackage{amssymb}
\usepackage{graphicx}
\usepackage{fancyhdr}

\pagestyle{fancy}

\rhead{{\bf Assigned:} Wednesday, February 11, 2014 \\{\bf Due:} Wednesday, February 18, 2014}

\printanswers
%\noprintanswers
\newcommand{\ds}{\displaystyle}
\newcommand{\lm}{\lim\limits}
\newtheorem{Definition}{Definition}

\begin{document}
\vspace{100mm}
\begin{center} \Large
MTH 371: Homework 3 \\ Taylor Series \normalsize
\end{center}
\ \\
\noindent GENERAL HOMEWORK GUIDELINES: 
\begin{itemize}
\item On the very first page of your homework, provide your name, date, and homework number.\vspace{-2mm}
\item Homework will be graded in part on neatness, organization, and completeness of solutions. Multiple pages MUST BE STAPLED. \vspace{-2mm}
\item Attach all Scilab code, output, and plots to the page \emph{immediately following} each problem. \vspace{-2mm}
\item Clearly label all plots (title, $x$-axis, $y$-axis, legend). Use the ``subplot" when needed
\end{itemize}


\begin{questions}


%%%%%%%%%%%%%%%%%%%%%%%%%%%%%%%%%%%%%%%%%%%%%%%%%%%%%%%%%%%%%%%%%%%%%
\question Compute by hand the first 5 terms in the Taylor series (constant, linear, quadratic, cubic, and quartic terms) for the following functions.
\begin{parts}
\part $f(x) = 3\tan(x)$, \quad about the point $x = \pi/4$.
\part $f(x) = e^{\cos(x)}$, \quad about the point $x=0$.
\end{parts}

%%%%%%%%%%%%%%%%%%%%%%%%%%%%%%%%%%%%%%%%%%%%%%%%%%%%%%%%%%%%%%%%%%%%%
\question Using the both parts (a) and (b) from problem 1, make a single Scilab plot which contains all of the following. 
\begin{parts}
\part a graph of $f(x)$ versus $x$ for $x\in(-3,3)$,
\part a graph of $P_2(x)$.
\part a graph of $P_4(x)$.
\part a title, $x$-axis label, $y$-axis label, and a legend.
\end{parts}
What role does the center of the Taylor series play with these graphs?

%%%%%%%%%%%%%%%%%%%%%%%%%%%%%%%%%%%%%%%%%%%%%%%%%%%%%%%%%%%%%%%%%%%%%
\question Find the second Taylor polynomial $P_2(x)$ for the function $f(x)=e^x\cos(x)$ about $x_0=0$.
\begin{parts}
\part  Using Taylor's theorem from class, find an upper bound for $|f(0.5)-P_2(0.5)|$. Compare this bound to the true error $|f(0.5)-P_2(0.5)|$.
\part Find a bound for the error $|f(x)-P_2(x)|$ where $P_2$ approximates $f$ on interval $[0,1]$.
\part Plot $f$ and $P_2$ on $[0,1]$ in Scilab. Approximately, where does the maximum error occur?
\part Approximate $\int_0^1 f(x)~dx$ by $\int_0^1 P_2(x)~dx$.
\part Find an upper bound for the error in (d) by using $\int_0^1 |R_2(x)|~dx$. Compare this bound to the true error.
\end{parts}

%%%%%%%%%%%%%%%%%%%%%%%%%%%%%%%%%%%%%%%%%%%%%%%%%%%%%%%%%%%%%%%%%%%%%
\question Give the Taylor series for $f(x)=x^3-2x^2+4x-1$ using center $x=2$. Discuss how this series compares to the original function. 

%%%%%%%%%%%%%%%%%%%%%%%%%%%%%%%%%%%%%%%%%%%%%%%%%%%%%%%%%%%%%%%%%%%%%
\question 
\begin{parts}
\part Derive the Maclaurin series for $f(x)=\cos(x)$. What is it's radius of convergence and why?
\part How many terms are needed in the series from part (a) to compute $\cos(x)$ for $|x|<0.5$ accurate to 12 decimal places? Verify your results using Scilab.
\end{parts}

\pagebreak
%%%%%%%%%%%%%%%%%%%%%%%%%%%%%%%%%%%%%%%%%%%%%%%%%%%%%%%%%%%%%%%%%%%%%
\question An alternating series is of the form
\[
S= \sum_{k=1}^{\infty} (-1)^{k-1}b_k = b_1 - b_2 + b_3 - b_4 + \dots, \quad b_k>0
\]
If both $b_{k+1} \leq b_k$ for all $k$ and also $\ds \lim_{k\rightarrow \infty} b_k = 0$, then the series is convergent. In addition, for the $n$th partial sum $\ds S_n =  \sum_{k=1}^{n} (-1)^{k-1}b_k$, we have the following error formula.
\[
|S-S_n| \leq b_{n+1}
\] 
Use this result to answer the following.
\begin{parts}
\part If you use the Maclaurin series for $\sin(x)$ to approximate $\sin(1)$ to within error $0.5 \times 10^{-6}$, how many terms in the series are needed? Use Scilab to check your answer.
\part Derive a series for the natural logarithm $\ln(x)$ with center $x=-1$. How many terms of this series are needed to compute an approximation to $\ln(2)$ within error $0.5 \times 10^{-6}$?
\end{parts}


\end{questions}

\end{document} 