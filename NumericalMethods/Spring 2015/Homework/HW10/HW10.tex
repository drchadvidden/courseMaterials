\documentclass[addpoints, 11pt]{exam}

\usepackage{amsmath}
\usepackage{amssymb}
\usepackage{graphicx}
\usepackage{fancyhdr}

\pagestyle{fancy}

\rhead{{\bf Assigned:} Friday, April 10, 2015 \\{\bf Due:} Friday, April 17, 2015}

\printanswers
%\noprintanswers
\newcommand{\ds}{\displaystyle}
\newcommand{\lm}{\lim\limits}
\newtheorem{Definition}{Definition}

\begin{document}
\vspace{100mm}
\begin{center} \Large
MTH 371: Homework 10 \\ Hermite and Spline Interpolation \normalsize
\end{center}
\ \\
\noindent GENERAL HOMEWORK GUIDELINES: 
\begin{itemize}
\item On the very first page of your homework, provide your name, date, and homework number.\vspace{-2mm}
\item Homework will be graded in part on neatness, organization, and completeness of solutions. Multiple pages MUST BE STAPLED. \vspace{-2mm}
\item Attach all Scilab code, output, and plots to the page immediately following each problem. Also, clearly indicate the problem they correspond to.
%\item Clearly label all plots (title, $x$-axis, $y$-axis, legend). Use the \verb1subplot 1command when comparing 2 or more plots.
\end{itemize}


\begin{questions}

%%%%%%%%%%%%%%%%%%%%%%%%%%%%%%%%%%%%%%%%%%%%%%%%%%%%%%%%%%%%%%%%%%%%%
\question Use Scilab to fit various piecewise polynomials to the Runge function $f(x) = \frac{1}{1+x^2}$ on the interval $[-5,5]$ using 13 equally spaced points.
\begin{parts}
\part Use either your \verb1LagrangeInt.sci1 or \verb1NewtonInt.sci1 to find the degree 12 polynomial through these points. Plot $f$ and this polynomial on the same graph.
\part Write your own code to find the piecewise linear interpolant of $f$ through these points. Plot $f$ and this piecewise polynomial on the same graph.
\part Write your own code to find the piecewise cubic Hermite interpolant of $f$ through these points. Plot $f$ and this Hermite interpolant on the same graph.
\part Use the Scilab \verb1splin1 command to find the natural cubic spline of $f$ through these points. Plot $f$ and this spline interpolant on the same graph.
\end{parts}
Feel free to use the \verb1subplot1 command to have all 4 of these plots on the same figure. Include your Scilab code with your submission.

%%%%%%%%%%%%%%%%%%%%%%%%%%%%%%%%%%%%%%%%%%%%%%%%%%%%%%%%%%%%%%%%%%%%%
\question Determine the piecewise polynomial function
$$
P(x) = 
\begin{cases}
P_1(x), & \text{if } 0\leq x \leq 1 \\
P_2(x), & \text{if } 1\leq x \leq 2
\end{cases}
$$
such that
\begin{parts}
\part $P_1(x)$ is linear,
\part $P_2(x)$ is quadratic,
\part $P(x)$ and $P'(x)$ are continuous at $x=1$,
\part $P(0)=1$, $P(1)=-1$, and $P(2)=0$.
\end{parts}
Graph this function.

%%%%%%%%%%%%%%%%%%%%%%%%%%%%%%%%%%%%%%%%%%%%%%%%%%%%%%%%%%%%%%%%%%%%%
\question Let $f$ be a given function satisfying $f(0)=1$, $f(1)=2$, and $f(2)=0$. A \emph{quadratic} spline interpolant $r(x)$ is defined as a piecewise quadratic that interpolates $f$ at the nodes \\ $x_0=0, x_1=1, x_2=2$ and whose first derivative is continuous throughout the interval. Find the quadratic spline interpolant of $f$ which also satisfies $r'(0)=0$. [Hint: Start from the left subinterval.] Plot your result in Scilab and compare to the cubic spline resulting from the Scilab \verb1splin1 command.

%%%%%%%%%%%%%%%%%%%%%%%%%%%%%%%%%%%%%%%%%%%%%%%%%%%%%%%%%%%%%%%%%%%%%
\question Let $f(x) = x^2(x-1)^2(x-2)^2(x-3)^2$. What is the piecewise cubic Hermite interpolant of $f$ on the grid $x_0=0, x_1=1, x_2=2, x_3=3$? Let $g(x)=ax^3+bx^2+cx+d$ for some parameters $a,b,c,d$. What is the piecewise cubic Hermite interpolant of $g$ on the same grid? [Hint: You should not need to do any arithmetic or use any formulas for either of these problems.] 

%%%%%%%%%%%%%%%%%%%%%%%%%%%%%%%%%%%%%%%%%%%%%%%%%%%%%%%%%%%%%%%%%%%%%
\question Show that the following function is a natural cubic spline through the points $(0,1), (1,1), (2,0),$ and $(3,10)$.
$$
s(x) =
\begin{cases}
1+x-x^3, & \text{if } 0 \leq x < 1 \\
1-2(x-1)-3(x-1)^2+4(x-1)^3, & \text{if } 1 \leq x < 2 \\
4(x-2)+9(x-2)^2-3(x-2)^3, & \text{if } 2 \leq x \leq 3
\end{cases}
$$



\end{questions}
\end{document} 
