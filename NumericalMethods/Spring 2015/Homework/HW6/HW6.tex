\documentclass[addpoints, 11pt]{exam}

\usepackage{amsmath}
\usepackage{amssymb}
\usepackage{graphicx}
\usepackage{fancyhdr}

\pagestyle{fancy}

\rhead{{\bf Assigned:} Wednesday, March 4, 2015 \\{\bf Due:} Wednesday, March 11, 2015}

\printanswers
%\noprintanswers
\newcommand{\ds}{\displaystyle}
\newcommand{\lm}{\lim\limits}
\newtheorem{Definition}{Definition}

\begin{document}
\vspace{100mm}
\begin{center} \Large
MTH 371: Homework 6 \\ Gaussian Elimination and Pivoting \normalsize
\end{center}
\ \\
\noindent GENERAL HOMEWORK GUIDELINES: 
\begin{itemize}
\item On the very first page of your homework, provide your name, date, and homework number.\vspace{-2mm}
\item Homework will be graded in part on neatness, organization, and completeness of solutions. Multiple pages MUST BE STAPLED. \vspace{-2mm}
\item Attach all Scilab code, output, and plots to the page immediately following each problem. Also, clearly indicate the problem they correspond to.
%\item Clearly label all plots (title, $x$-axis, $y$-axis, legend). Use the \verb1subplot 1command when comparing 2 or more plots.
\end{itemize}


\begin{questions}


%%%%%%%%%%%%%%%%%%%%%%%%%%%%%%%%%%%%%%%%%%%%%%%%%%%%%%%%%%%%%%%%%%%%%
\question Consider the linear system $A\vec{x}=\vec{b}$ where
$$
A= \left[
\begin{array}{cccc}
6 & -2 & 2 & 4 \\
12 & -8 & 6 & 10 \\
3 & -13 & 9 & 3 \\
-6 & 4 & 1 & -18
\end{array} \right], \quad
\vec{b} = \left[
\begin{array}{c}
16 \\
26 \\
-19 \\
-34
\end{array} \right]
$$
Find the $LU$ factorization of matrix $A$ and use it to solve $A\vec{x}=\vec{b}$. That is, write \\ $A\vec{x}=L(U\vec{x})=\vec{b}$, then solve $L\vec{y}=\vec{b}$ and next $U\vec{x}=\vec{y}$. Show this by hand and clearly show every step of the construction of $L$ and $U$ as well as all steps for forward and backwards substitution. 

%%%%%%%%%%%%%%%%%%%%%%%%%%%%%%%%%%%%%%%%%%%%%%%%%%%%%%%%%%%%%%%%%%%%%
\question Consider the linear system $A\vec{x}=\vec{b}$ for 
$$
A= \left[
\begin{array}{cccc}
2 & 4 & -2 & -2 \\
1 & 2 & 4 & -3 \\
-3 & -3 & 8 & -2 \\
-1 & 1 & 6 & -3
\end{array} \right], \quad
\vec{b} = \left[
\begin{array}{c}
-4 \\
5 \\
7 \\
7
\end{array} \right]
$$
\begin{parts}
\part Using partial pivoting, compute the $LU$ decomposition of the matrix $A$ with partial pivoting at each step resulting in $PA=LU$. Show all steps of this decomposition by hand. 
\part Verify $PA=LU$ by matrix multiplication.
\part Use the decomposition from (a) to solve $A\vec{x}=\vec{b}$.
\end{parts}

%%%%%%%%%%%%%%%%%%%%%%%%%%%%%%%%%%%%%%%%%%%%%%%%%%%%%%%%%%%%%%%%%%%%%
\question Write Scilab Functions for Gaussian elimination via $LU$ decomposition for the general $n$-dimensional system $A\vec{x}=\vec{b}$. To do this, create functions which implement forward substitution, backward substitution, $LU$ decomposition without pivoting, and $LU$ decomposition with partial pivoting. Functions should be written so they can be called in Scilab by typing:
\begin{parts}
\part \verb1y = ForwardSubs(L,b)1 (forward substitution)
\part \verb1x = BackwardsSubs(U,y)1 (backwards substitution)
\part \verb1[L,U] = LU(A)1 ($LU$ decomposition, no pivoting)
\part \verb1[P,L,U] = PLU(A)1 ($LU$ decomposition with partial pivoting, $PA=LU$)
\end{parts}
All functions should be stored in a dedicated .sci file, so 4 files total. Include this code with your homework submission. You should refer to the Gaussian elimination code you created in last week's homework.

%%%%%%%%%%%%%%%%%%%%%%%%%%%%%%%%%%%%%%%%%%%%%%%%%%%%%%%%%%%%%%%%%%%%%
\question Use the code developed in problem 3 to verify your work from problem 1. Write a .sce file which calls the needed functions. Print the results generated by your code and include this .sce file with your homework submission.

%%%%%%%%%%%%%%%%%%%%%%%%%%%%%%%%%%%%%%%%%%%%%%%%%%%%%%%%%%%%%%%%%%%%%
\question Use the code developed in problem 3 to verify your work from problem 2. Write a .sce file which calls the needed functions. Print the results generated by your code and include this .sce file with your homework submission.

%%%%%%%%%%%%%%%%%%%%%%%%%%%%%%%%%%%%%%%%%%%%%%%%%%%%%%%%%%%%%%%%%%%%%
%\question Compute the number of floating-point operations (additions, subtractions, multiplications, and divisions) which are required for:
%\begin{parts}
%\part The product of a $(m\times n)$ matrix with a $(n\times p)$ matrix. 
%\part The original Gaussian elimination algorithm for a $(n\times n)$ matrix developed in class.
%\part The more efficient Gaussian elimination algorithm for a $(n \times n)$ matrix developed in class.
%\part Forward substitution on $L \vec{y} = \vec{b}$ for $L$ $(n\times n)$ resulting from $LU$ decomposition.
%\part Backwards substitution on $U \vec{y} = \vec{x}$ for $U$ $(n\times n)$ resulting from $LU$ decomposition.
%\end{parts}

%%%%%%%%%%%%%%%%%%%%%%%%%%%%%%%%%%%%%%%%%%%%%%%%%%%%%%%%%%%%%%%%%%%%%
\question The matrix factorization $LU=PA$ can be used to compute the determinant of $A$.
\begin{parts}
\part Explain why it must be true that $\det(L)\det(U)=\det(P)\det(A)$.
\part What is $\det(L), \det(U)$ and $\det(P)$ in general and why? Each of these should be easy to compute.
\part Modify your function of problem 3(d) \verb1[P,L,U,sig] = PLU(A)1 to also returns \verb1sig1 equal to +1 or -1 if \verb1P1 is an even or odd permutation.
\part Create a new function \verb1mydet(A)1 which calls function \verb1PLU(A)1 of part 6(c) and only uses entries of \verb1U1 and the value of \verb1sig1. Test this function on the matrix of problem 2. Compare your result to the built in Scilab determinant function.
\end{parts}


%%%%%%%%%%%%%%%%%%%%%%%%%%%%%%%%%%%%%%%%%%%%%%%%%%%%%%%%%%%%%%%%%%%%%
\question 
\begin{parts}
\part The inverse of a matrix $A$ can be defined s the matrix $X$ whose columns $\vec{x}_j$ solve the equations
\[
A\vec{x}_j = \vec{e}_j
\]
where $\vec{e}_j$ is the $j$th unit basis vector. Explain why this is true.
\part Create your own function \verb1myinv(A)1 which calls \verb1PLU(A)1 from problem 3 \emph{only once} and also utilizes \verb1BackwardsSubs(U,y)1.
\part Test your function on problems 1 and 2 and compare to the built in Scilab function.
\end{parts}


\end{questions}
\end{document} 