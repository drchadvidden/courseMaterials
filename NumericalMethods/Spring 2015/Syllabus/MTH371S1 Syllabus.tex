\documentclass [10pt]{article}

\usepackage{hyperref}

\setlength{\evensidemargin}{0.0cm}
\setlength{\oddsidemargin}{0.0cm}
\setlength{\topmargin}{-1.75cm}
%\setlength{\baselineskip}{20pt}
\setlength{\textwidth}{17cm}
\setlength{\textheight}{24.5cm}
\hoffset = -0.75cm
\pagestyle{empty}


\begin {document}
\begin {center}
\Large \bfseries MTH 371 \\
Introduction to Numerical Methods \normalfont \normalsize \\
Spring 2015 \\
MWF 1:10-2:05pm \\
111 Cowley Hall
\end {center}
\ \\
\noindent
\bfseries Instructor: \normalfont Chad Vidden \\
\bfseries Office: \normalfont 1009 Cowley Hall \\
\bfseries Office Hours: \normalfont Mon 2:10-3pm, Tues 11-12am, 1-2pm, Thurs 11-12am, 1-3pm \emph{or by appointment} \\
\bfseries Email: \normalfont cvidden@uwlax.edu \\
\bfseries Course Website: \normalfont D2L: \url{http://www.uwlax.edu/d2l/}\\
\bfseries Textbook: \normalfont \itshape Numerical Methods \normalfont by Burden and Faires, 4th Ed. \\
\bfseries Supplementary Textbook: \normalfont \itshape Numerical Mathematics and Computing \normalfont by Cheney and Kinkaid, 3rd Ed. \\
\ \\
\bfseries Computing Software: \normalfont Scilab \vspace{-3mm} 
\begin{itemize}
\setlength{\itemsep}{1pt}
\setlength{\parskip}{0pt}
\setlength{\parsep}{0pt}
\item Scilab is free and open source software for numerical computation (similar to Matlab) providing a powerful computing environment for engineering and scientific applications. 
\item For more information and to download this software, visit \url{www.scilab.org}.
\end{itemize} 
\noindent
\bfseries Course Description: \normalfont Techniques devised for use with computing machinery are applied to problems such as: solving non-linear equations and linear systems, curve-fitting and function approximation, numerical integration.  Prerequisites: MTH 309 and CS 120. \\
\ \\
\bfseries Course Contents: \normalfont   \vspace{-3mm} 
\begin{itemize}
\setlength{\itemsep}{1pt}
\setlength{\parskip}{0pt}
\setlength{\parsep}{0pt}
\item Chapter 1: Introduction (brief Calculus review, Taylor series, rates of convergence, floating point numbers) 
\item Chapter 2: Root finding methods (bisection, false position, secant method, Newton's method)
\item Chapter 6: Direct methods for solving linear systems (Gaussian elimination, LU factorization, Newton's method for nonlinear systems)
\item Chapter 3: Interpolation and polynomial approx (Lagrange form, Newton form, splines, least squares)
\item Chapter 4: Numerical differentiation and integration (finite differences, Richardson extrapolation, Newton-Cotes, Gaussian quadrature, Romberg integration, adaptive quadrature, improper integrals)
\item Chapter 5: Initial-value problems for ODEs (Euler's method, Taylor methods, Runge-Kutta methods, multi-step methods, absolute stability)
\end{itemize}
\noindent
\bfseries Learning Outcomes: \normalfont By the end of this course, the student should be able to: \vspace{-3mm} 
\begin {itemize}
\setlength{\itemsep}{1pt}
\setlength{\parskip}{0pt}
\setlength{\parsep}{0pt}
\item develop numerical methods for approximately solving problems from continuous mathematics.
\item analyze these methods in terms of accuracy, stability, and efficiency.
\item implement and experiment with these methods in a computer language.
\end{itemize}
\noindent
\bfseries Course Policy: \normalfont \vspace{-3mm} 
\begin {itemize}
\setlength{\itemsep}{1pt}
\setlength{\parskip}{0pt}
\setlength{\parsep}{0pt}
\item Attendance will not be enforced, though it is an essential part of this course. Be at class on time every day and be prepared to ask questions and participate. If you do miss class, check the course website for the material that you missed.
\item Students are encouraged to bring their laptops to class. Often, computing demonstrations will be given during lecture, and it may be useful to follow along and experiment yourself. Scripts for lecture will be provided ahead of time on the course website.
\item It is important that you read the textbook carefully for understanding. We will not be able to cover all examples and ideas in the textbook in class, but you are responsible for the content in the textbook. 
\item Make-up exams will be allowed only in extreme circumstances and will require legitimate proof of excuse. Contact me as soon as possible if something like this should occur, and we will work out a solution. Note, I must agree that your circumstance warrants making up the exam beforehand.
\item The use of cell phones, mp3 players, or anything else that could be distracting to others during class is not allowed. Be sure to turn off or silence cell phones before class begins.
\item All announcements, course materials, and grades will be listed on the course D2L website.
\end {itemize}
\pagebreak
\noindent
\bfseries Grading Policy: \normalfont
Your overall grade for this course will be based on scores earned from homework, quizzes, a writing assignment, four regular in class exams, and one comprehensive final exam. Each portion of your grade is worth the following points. \vspace{-3mm} 
\begin {itemize}
\setlength{\itemsep}{1pt}
\setlength{\parskip}{0pt}
\setlength{\parsep}{0pt}
\item Homework: scaled to 150 points
\item Group Projects: (20 points) x (3 projects) =  scaled to 100 points
\item Quizzes: (10 points) x (10 quizzes) = 100 points
\item Two In-Class Exams: (100 points) x (2 exams) = 200 points
\item Final Exam: 120 points
\end {itemize}
By taking the percentage of points you earn in this course, your final letter grade will be no worse than what is listed below. \\
100-92, A; 92-88, A/B; 88-82, B; 82-78, B/C; 78-70, C; 70-60, D; 60-0, F \\ 
\ \\
\bfseries Homework: \normalfont Homework will be assigned and due on every Wednesday. You are encouraged to work in groups on homework assignments, but you submission should be written independently of your own thoughts and ideas.\\
\ \\
\bfseries Group Projects: \normalfont 5 group projects will be assigned throughout the semester. Details will be given in class. \\
\ \\
\bfseries Quizzes: \normalfont At the end of class on Wednesday of every week, a short quiz will be given. The purpose of the quizzes is to test understanding of basic ideas discussed in the homework for that week. Makeup quizzes will not be given. Your two lowest quiz scores will be dropped for the final grade calculation. \\
\ \\
\bfseries Exams: \normalfont Two in-class exams will be given throughout the semester along with one final exam. All exams are comprehensive. Notes, and textbooks will not be allowed. Makeup exams will not be given.  \vspace{-3mm} 
\begin {itemize}
\setlength{\itemsep}{1pt}
\setlength{\parskip}{0pt}
\setlength{\parsep}{0pt}
\item Exam 1: Friday, March 6
\item Exam 2: Friday, April 17
\item Final Exam: Friday, May 15, 12:15-2:15 pm, location TBA.
\end{itemize}
\bfseries Cheating Policy: \normalfont Cheating will not be tolerated. If you are caught cheating on a quiz, project, or exam this will at the very least result in a zero for that assessment. Further action will be taken in severe circumstances. See \url{ http://www.uwlax.edu/StudentLife/academic_misconduct.htm#14.03} for more information.\\
\ \\
\bfseries Dropping the Course: \normalfont If personal circumstances or low grades necessitate dropping this course, it is your responsibility to contact your advisor and complete the proper forms. The last day to drop is Friday, March 28. Talk to your advisor and instructor before dropping any course. \\
\ \\
\bfseries Students with Disabilities: \normalfont Any student with a documented disability (e.g., physical, sensory, psychological, learning disability, AD/HD, or are a current or prior military service member with wounded warrior status) who needs to arrange reasonable academic accommodations must contact Disability Resource Services (165 Murphy Library, (608) 785-6900) at the beginning of the semester. In addition to registering with Disability Resource Services, it is the student’s responsibility to discuss their needs with the instructor in a timely manner. \\
\ \\
\bfseries Eagle Alert: \normalfont This class will be participating in the “Eagle Alert” system through WINGS. The Early Alert system is designed to promote student success. If I notice that you are experiencing difficulties early in the semester (e.g., low assignment scores or poor attendance), I may enter feedback into the program and you will receive an email indicating that feedback has been left.  I may also enter positive feedback encouraging you to think about additional opportunities. The link in the email will take you to WINGS where you will login to see the feedback. I encourage you to meet with me and use one or more of several helpful campus resources listed here \url{http://www.uwlax.edu/studentsuccess/}. \\
\ \\
Note, this syllabus is subject to change at any time. Changes will be announced in class, and an updated syllabus will be posted on the course website.

\vfill

\end {document} 