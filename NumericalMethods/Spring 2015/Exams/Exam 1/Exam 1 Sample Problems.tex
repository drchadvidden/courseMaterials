\documentclass[addpoints, 11pt]{exam}

\usepackage{amsmath}
\usepackage{graphicx}

\oddsidemargin=0in
\evensidemargin=0in
\textwidth=6.3in
\topmargin=-0.5in
\textheight=9in
\parindent=0in
\pagestyle{empty}
\thispagestyle{empty}

%\printanswers
\noprintanswers

\newtheorem{Definition}{Definition}
\newcommand{\ds}{\displaystyle}

\begin{document}

\begin{center}
\Large \bfseries Exam 1 Sample Problems \\
Math 371 \\ \normalsize
\end{center}

\begin{questions}
\question Compute the rates of convergence of the following limits.
\begin{parts}
\part $\ds \lim_{h\rightarrow 0} \frac{\sin(h)-h\cos(h)}{h}$
\part $\ds \lim_{h\rightarrow 0} \frac{\sin(h)}{h}$
\part $\ds \lim_{h\rightarrow 0} \frac{1-e^h}{h}$
\part $\ds \lim_{n\rightarrow \infty} \frac{n+5}{n^2}$
\part $\ds \lim_{n\rightarrow \infty} \frac{2n^2}{n^2+1}$
\part $\ds \lim_{n\rightarrow \infty} \sin(1/n)$
\end{parts}
\question What empirical evidence usually suggests quadratic convergence?
\question Give the Taylor series expansion for function $f(x)$ about $x=a$.
\question State Taylor's theorem.
\question Compute by the hand the Taylor series expansion for 
\begin{parts}
\part $f(x)=e^x$ about $x=0$
\part $f(x)=e^{x^2}+e^{2x}$ about $x=0$
\part $f(x)=\sin(x)$ about $x=0$
\part $f(x)=\cos(x)$ about $x=0$
\part $f(x)=\ln(x+1)$ about $x=0$
\part $f(x)=\arctan(x)$ about $x=0$
\part $f(x)x^4-3x^2+1=$ about $x=1$
\part $f(x)=\sqrt{x}$ about $x=16$
\end{parts}
\question What is the maximum absolue error posible when using the approximation
$$
\sin(x) \approx x - \frac{x^3}{3!}+\frac{x^5}{5!}
$$
for $-0.3 \leq x \leq 0.3$? For what $x$ values is the approximation accurate to within $0.00005$?
\question Give the degree 2 Taylor polynomial $P(x)$ for $f(x)=x^3$ about $x=1$. Find the value of $\xi$ in $[1,3]$ such that $f(3)=P(3)+\frac{f''(\xi)}{6}(3-1)^3$.
\question Give absolute and relative errors in approximating $\frac{1}{3}$ with $0.33$.
\question As an approximation of $1.23456$, how many significant digits are there in $x = 1.237$? \pagebreak
\question Derive a formula for the following rootfinding methods. Give a complete explanation.
\begin{parts}
\part Bisection method
\part Secant method
\part Method of false position
\part Newtons method
\end{parts}
\question Show that the Bisection method on the interval $[a,b]$ requires $n>\log_2((b-a)/TOL)$ to ensure absolute error less than some tolerance $TOL$. 
\question Apply the result of the previous problem to show that $n=14$ steps are required for the Bisection method to approximate the root of $f(x)=x^3+4x^2-10$ on $[1,2]$ within absolute error less than $10^{-4}$.
\question For the rootfinding problem $f(x)=e^x-5x=0$, write down two equivalent fixed point problems.
\question For the following fixed point problem $x=g(x)=\frac{1}{2}\left(x+\frac{3}{x}\right)$, write down an equivalent rootfinding problems.
\question State the Fixed Point Theorem from class, both the linear and quadratic convergence cases.
\question Let $g(x)=\frac{1}{10}(x^2+x+8)$.
\begin{parts}
\part Find the smallest positive fixed point of $g$.
\part Using the Fixed Point Theorem from class, show that starting with any $x_0 \in [0,4]$, the sequence $x_n = g(x_{n-1})$ will converge to the smallest fixed point of $g$.
\end{parts}
\question Suppose that $g(x)$ has a fixed point $r$ in $[a,b]$ and that $|g'(x)|\leq K<1$ on $[a,b]$ for some constant $K$. Prove that starting with any $x_0 \in [a,b]$, the sequence $x_n=g(x_{n-1})$ converges to a fixed point of $g$. 
\question Let $g(x)=\frac{x(2+x)}{-1+4x}$.
\begin{parts}
\part Find the two fixed points of $g$.
\part To which fixed point will convergence be quadratic? 
\end{parts}
\question Use the Fixed Point Theorem from class to show that Newton's method has quadratic rate of convergence for initial guess close enough to the root $r$ provided $f(r)=0$ and $f'(r)\neq 0$.
\question Consider the following system, $A\vec{x}=\vec{b}$.
$$
\left[
\begin{array}{ccc}
1 & -1 & 2 \\
-2 & 1 & -1 \\
4 & -1 & 2
\end{array}
\right] 
\left[
\begin{array}{c}
x_1 \\
x_2 \\
x_3
\end{array}
\right]
= \left[
\begin{array}{c}
-2 \\ 
2 \\
-1
\end{array}
\right]
$$ 
\begin{parts}
\part Perform regular Gaussian elimination to solve this system.
\part Find the $LU$ decomposition of matrix $A$ without pivoting, and use this decomposition to solve this system.
\part Perform Gaussian elimination with partial pivoting to solve this system.
\part Find the $LU$ decomposition of matrix $A$ with pivoting, and use this decomposition to solve this system.
\end{parts}
\question Problem 1 on page 265 of the text.
\question Compute the number of floating-point operations (additions, subtractions, multiplications, and divisions) which are required for:
\begin{parts}
\part The product of a $(m\times n)$ matrix with a $(n\times p)$ matrix. 
\part Forward substitution on $L \vec{y} = \vec{b}$ for $L$ $(n\times n)$ resulting from $LU$ decomposition.
\part Backwards substitution on $U \vec{y} = \vec{x}$ for $U$ $(n\times n)$ resulting from $LU$ decomposition.
\end{parts}
\question What does it mean for a linear system to be ill-conditioned? What is the condition number of a matrix, and roughly, where does it come from?
\question Compute the least squares solution to the following overdetermined system.
$$
\left[
\begin{array}{cc}
2 & -1  \\
1 & 2 \\
1 & 1
\end{array}
\right] 
\left[
\begin{array}{c}
x_1 \\
x_2 
\end{array}
\right]
= \left[
\begin{array}{c}
2 \\ 
1 \\
4
\end{array}
\right]
$$
\question Construct the Lagrange iterpolation polynomial for function $f(x) = 2^x$ using nodes $x_0=0$, $x_1=1$, and $x_2=3$. What is the absolute error at $x=1$?
\question State the theorem giving the polynomial interpolation error.
\end{questions}

\end{document} 