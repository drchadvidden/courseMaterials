\documentclass[addpoints, 11pt]{exam}

\usepackage{amsmath}
\usepackage{graphicx}
\usepackage{multicol}

\oddsidemargin=0in
\evensidemargin=0in
\textwidth=6.3in
\topmargin=-0.5in
\textheight=9in
\parindent=0in
\pagestyle{empty}
\thispagestyle{empty}

%\printanswers
\noprintanswers

\newtheorem{Definition}{Definition}
\newcommand{\ds}{\displaystyle}
\newcommand{\ul}{\underline}

\begin{document}

Math 371 \\
Exam 1, \ \ March 13, 2015
\hspace{1.9in}
{Name:} {\underline {\hspace{2.15in}}}
\begin{center}
\fbox{\fbox{\parbox{6.2in}{
You have until the end of the hour to complete this exam. Show all work, justify your solutions
completely, simplify as much as possible. The only materials you should have on your
desk are this exam and a pencil. If you have any questions, be sure to ask for clarification.
 }}}
\end{center}
\begin{questions}


%%%%%%%%%%%%%%%%%%%%%%%%%%%%%%%%%%%%%%%%%%%%%%%%%%%%%%%%%%%%%%%%%%%%%%%
\question[10] 
\begin{parts}
\part Convert the binary number 1011.11 to decimal. \vfill
\part Convert the decimal number 27 to binary. \vfill
\part Add the above two numbers using binary arithmetic. \vfill
\end{parts}

%%%%%%%%%%%%%%%%%%%%%%%%%%%%%%%%%%%%%%%%%%%%%%%%%%%%%%%%%%%%%%%%%%%%%%%
\question[5] Consider a number system which can only store numbers of the form $\pm 1.b_1b_2 \times 2^E$ for $E=-1, 0, 1$. Exactly, what is machine epsilon in this system and why? \vfill

%%%%%%%%%%%%%%%%%%%%%%%%%%%%%%%%%%%%%%%%%%%%%%%%%%%%%%%%%%%%%%%%%%%%%%%
\question[10] 
\begin{parts}
\part Compute by hand the 4th degree Taylor polynomial $P(x)$ for function $f(x)=\sin(x)$ around $a=0$. \vfill
\part Use Taylor's theorem to compute the maximum error of $|f(x)-P(x)|$ on $-0.3\leq x \leq 0.3$.  \vfill
\end{parts}

\pagebreak

%%%%%%%%%%%%%%%%%%%%%%%%%%%%%%%%%%%%%%%%%%%%%%%%%%%%%%%%%%%%%%%%%%%%%%%
\question[10] If you use the $n$th degree Taylor polynomial of $f(x)=e^x$ centered at $x_0=0$ to approximate $e$, what should $n$ be to guarantee accuracy within absolute error $10^{-9}$.  \vfill \vfill

%%%%%%%%%%%%%%%%%%%%%%%%%%%%%%%%%%%%%%%%%%%%%%%%%%%%%%%%%%%%%%%%%%%%%%%
\question[10]
\begin{parts}
\part Let the rootfinding problem $f(x)=0$ have solution $x=p$. Also, let $p_n$ be the $n$th term found by the bisection method. Show that if we want the absolute error less than some error tolerance $TOL$, i.e. $|p-p_n|<TOL$, we need $\ds n > \log_2\left(\frac{b-a}{TOL}\right)$.  \vfill \vfill \vfill
\part If the bisection method converges, what rate does it converge at? What does this mean precisely? \vfill
\end{parts}

\pagebreak


%%%%%%%%%%%%%%%%%%%%%%%%%%%%%%%%%%%%%%%%%%%%%%%%%%%%%%%%%%%%%%%%%%%%%%%
\question[12] Consider the fixed point problem $x=g(x)=\frac{1}{2}\left(x+\frac{3}{x}\right)$.
\begin{parts}
\part State a fixed-point iteration for this problem. \vfill
\part Rewrite this fixed point problem as a root-finding problem. \vfill
\part State Newton's method for the root-finding problem in (b). \vfill
\end{parts}

%%%%%%%%%%%%%%%%%%%%%%%%%%%%%%%%%%%%%%%%%%%%%%%%%%%%%%%%%%%%%%%%%%%%%%%
\question[13] Let $g(x)=\frac{1}{10}(x^2+x+8)$.
\begin{parts}
\part Find the smallest positive fixed point of $g$.  \vfill
\part Using the Fixed Point Theorem from class, show that starting with any $x_0 \in [0,4]$, the sequence $x_n = g(x_{n-1})$ will converge to the smallest fixed point of $g$.  \vfill \vfill
\part What is the rate of convergence of the fixed point iteration in (b)? Can it be quadratic?  \vfill
\end{parts}

\pagebreak

%%%%%%%%%%%%%%%%%%%%%%%%%%%%%%%%%%%%%%%%%%%%%%%%%%%%%%%%%%%%%%%%%%%%%%%
\question[20] Consider the following system, $A\vec{x}=\vec{b}$.
$$
\left[
\begin{array}{ccc}
1 & -1 & 2 \\
-2 & 1 & -1 \\
4 & -1 & 2
\end{array}
\right] 
\left[
\begin{array}{c}
x_1 \\
x_2 \\
x_3
\end{array}
\right]
= \left[
\begin{array}{c}
-2 \\ 
2 \\
-1
\end{array}
\right]
$$ 
\begin{parts}
\part Perform Gaussian elimination WITHOUT pivoting to solve this system. Use an augmented matrix and show all steps.  \vfill
\part Find the $LU$ decomposition of matrix $A$ without pivoting, and use this decomposition to solve this system. Feel free to use work from part (a).  \vfill
\end{parts}

%%%%%%%%%%%%%%%%%%%%%%%%%%%%%%%%%%%%%%%%%%%%%%%%%%%%%%%%%%%%%%%%%%%%%%%%%%%%
\pointname{ extra credit point}
\question[1] $\pi$-day bonus! State $\pi$ correct to 6 decimal digits.

\end{questions}
\end{document} 