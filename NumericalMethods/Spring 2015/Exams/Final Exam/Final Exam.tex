\documentclass[addpoints, 11pt]{exam}

\usepackage{amsmath}
\usepackage{graphicx}
\usepackage{multicol}

\oddsidemargin=0in
\evensidemargin=0in
\textwidth=6.3in
\topmargin=-0.5in
\textheight=9in
\parindent=0in
\pagestyle{empty}
\thispagestyle{empty}

%\printanswers
\noprintanswers

\newtheorem{Definition}{Definition}
\newcommand{\ds}{\displaystyle}
\newcommand{\ul}{\underline}

\begin{document}

Math 371 \\
Final Exam, \ \ May 13, 2014
\hspace{1.9in}
{Name:} {\underline {\hspace{2.15in}}}
\begin{center}
\fbox{\fbox{\parbox{6.2in}{
You have two hours to complete this exam. Show all work, justify your solutions completely, simplify as much as possible. The only materials you should have on your
desk are this exam and a pencil. If you have any questions, be sure to ask for clarification.
 }}}
\end{center}
\begin{questions}


%%%%%%%%%%%%%%%%%%%%%%%%%%%%%%%%%%%%%%%%%%%%%%%%%%%%%%%%%%%%%%%%%%%%%
%% POLYNOMIAL INTERPOLATION FORMULA
%%%%%%%%%%%%%%%%%%%%%%%%%%%%%%%%%%%%%%%%%%%%%%%%%%%%%%%%%%%%%%%%%%%%%
\question[12] 
\begin{parts}
\part State the polynomial interpolation error theorem. \vfill
\part Prove the polynomial interpolation error theorem. \vfill \vfill \vfill \vfill \vfill
\end{parts}
\pagebreak

%%%%%%%%%%%%%%%%%%%%%%%%%%%%%%%%%%%%%%%%%%%%%%%%%%%%%%%%%%%%%%%%%%%%%
%% ROOTFINDING AND FIXED POINT ITERATION
%%%%%%%%%%%%%%%%%%%%%%%%%%%%%%%%%%%%%%%%%%%%%%%%%%%%%%%%%%%%%%%%%%%%%
\question[12] Let function $g(x)=x^2(2-x)$.
\begin{parts}
\part Complete the following statement. A sequence $x_n$ converges to number $L$ quadratically if \vfill
\part Write down a fixed point iteration for finding the fixed points of $g$. \vfill
\part Find the fixed points of $g$ algebraically. \vfill
\part To which of the fixed points found in (c) will the iteration in (b) converge quadratically? Why? \vfill
\part Find a function $f$ with roots the same as the fixed points of $g$. \vfill
\part Write down Newton's method for finding the roots of function $f$ from part (e). \vfill
\part To which of the roots found of function $f$ will the iteration in (f) converge quadratically? Why? \vfill
\end{parts}
\pagebreak


%%%%%%%%%%%%%%%%%%%%%%%%%%%%%%%%%%%%%%%%%%%%%%%%%%%%%%%%%%%%%%%%%%%%%
%% GAUSSIAN ELIMINATION / LU DECOMPOSITION
%%%%%%%%%%%%%%%%%%%%%%%%%%%%%%%%%%%%%%%%%%%%%%%%%%%%%%%%%%%%%%%%%%%%%
\question[12] Find the $LU$ decomposition without partial pivoting for the following matrix. Show all steps!
$$
\left[
\begin{array} {ccc}
1 & 2 & 3 \\
3 & 1 & 4 \\
5 & 1 & 1
\end{array}  \right]
$$
\pagebreak


%%%%%%%%%%%%%%%%%%%%%%%%%%%%%%%%%%%%%%%%%%%%%%%%%%%%%%%%%%%%%%%%%%%%%
%% INTERPOLATION
%%%%%%%%%%%%%%%%%%%%%%%%%%%%%%%%%%%%%%%%%%%%%%%%%%%%%%%%%%%%%%%%%%%%%
\question[12] Prove that a polynomial interpolant of degree at most $n$ through the $(n+1)$ points  $\{(x_0, f(x_0)), (x_1, f(x_1)), \dots, (x_n, f(x_n)) \}$ must be unique.
\pagebreak


%%%%%%%%%%%%%%%%%%%%%%%%%%%%%%%%%%%%%%%%%%%%%%%%%%%%%%%%%%%%%%%%%%%%%%%
%% INTERPOLATION, NEWTON DIVIDED DIFFERENCES
%%%%%%%%%%%%%%%%%%%%%%%%%%%%%%%%%%%%%%%%%%%%%%%%%%%%%%%%%%%%%%%%%%%%%
\question[12]
\begin{parts}
\part A freshman calculus student was amazed to learn that the prime numbers have no predictable pattern. Mess with their head by using divided differences to produce a polynomial $P$ for which $P(n)$ gives the $n$th prime number for $n=1,2,3,4,5$. (Recall, the first prime number is 2). \vfill \vfill \vfill \vfill
\part Assuming the existence of such a miraculous function $f$ for which $f(n)$ gives the $n$th prime number, provided the error formula for $f(6)-P(6)$. \vfill
\end{parts}
\pagebreak


%%%%%%%%%%%%%%%%%%%%%%%%%%%%%%%%%%%%%%%%%%%%%%%%%%%%%%%%%%%%%%%%%%%%%
%% hermite / spline
%%%%%%%%%%%%%%%%%%%%%%%%%%%%%%%%%%%%%%%%%%%%%%%%%%%%%%%%%%%%%%%%%%%%%
\question[12] 
\begin{parts}
\part List all 8 conditions for $S(x)$ to be a natural cubic spline for a function $f$ through points $(-1,f(-1)),~(0,f(0)),~(1,f(1))$. \vfill
\part Do there exist constants $a,b,c,d$ such that the function
$$
S(x) =
\begin{cases}
ax^3 + x^2 + cx, & -1\leq x \leq 0 \\
bx^3 + x^2 + dx, & 0 \leq x \leq 1
\end{cases}
$$
is a natural cubic spline for $f(x)=|x|$ through points $(-1,1),~(0,0),~(1,1)$. \vfill
\end{parts}
\pagebreak


%%%%%%%%%%%%%%%%%%%%%%%%%%%%%%%%%%%%%%%%%%%%%%%%%%%%%%%%%%%%%%%%%%%%%%%
%% QUADRATURE RULES
%%%%%%%%%%%%%%%%%%%%%%%%%%%%%%%%%%%%%%%%%%%%%%%%%%%%%%%%%%%%%%%%%%%%%
\question[12] 
Recall that the error for the trapezoidal rule on one subinterval is given by
$$
\int_a^b f(x)~dx - T_1(a,b) = -\frac{(b-a)^3}{12} f''(\xi)
$$
where $T_1(a,b)$ denotes the trapezoidal rule you derived in part (a). Show that the error for the $n$ step composite trapezoidal rule approximation is given by
$$
\int_a^b f(x)~dx - T_n(a,b) = -\frac{(b-a)}{12} f''(\eta) h^2
$$
where $T_n(a,b)$ denotes the $n$ step composite trapezoidal rule and $h$ is the size of each subinterval.  
\pagebreak


%%%%%%%%%%%%%%%%%%%%%%%%%%%%%%%%%%%%%%%%%%%%%%%%%%%%%%%%%%%%%%%%%%%%%
%% gaussian quad
%%%%%%%%%%%%%%%%%%%%%%%%%%%%%%%%%%%%%%%%%%%%%%%%%%%%%%%%%%%%%%%%%%%%%
\question[12] 
\begin{parts}
\part Find the needed constants $A_0, A_1, x_0, x_1$ so that the quadrature rule of the form
$$
\int_{-1}^1 f(x)~dx \approx A_0 f(x_0) + A_1 f(x_1)
$$
is exact for degree 3 polynomials. What is the name of this quadrature rule? (Hint: Assume $A_0=1$.) \vfill
\part Use part (a) to find constants $B_0, B_1, t_0, t_1$ so that the quadrature rule of the form
$$
\int_{-1}^1 tg(t)~dt \approx B_0 g(t_0) + B_1 g(t_1)
$$
which is exact for degree 3 polynomials.  \vfill
\end{parts}
\pagebreak

%%%%%%%%%%%%%%%%%%%%%%%%%%%%%%%%%%%%%%%%%%%%%%%%%%%%%%%%%%%%%%%%%%%%%%%
%% numerical differentiation, centered difference
%%%%%%%%%%%%%%%%%%%%%%%%%%%%%%%%%%%%%%%%%%%%%%%%%%%%%%%%%%%%%%%%%%%%%
\question[12] 
\begin{parts}
\part Give the Taylor series expansion for $f(x-h)$ and $f(x+h)$ with center $x$ up til error $O(h^5)$. \vfill
\part Use part (a) to derive the central difference approximation formula for $f'(x)$. What is the order of the error? \vfill
\part Use your results in part (b) along with Richardson Extrapolation to create an approximation of $f'(x)$ which is $O(h^4)$. \vfill
\end{parts}
\pagebreak


%%%%%%%%%%%%%%%%%%%%%%%%%%%%%%%%%%%%%%%%%%%%%%%%%%%%%%%%%%%%%%%%%%%%%%%
%% EULER'S METHOD
%%%%%%%%%%%%%%%%%%%%%%%%%%%%%%%%%%%%%%%%%%%%%%%%%%%%%%%%%%%%%%%%%%%%%
\question[12] Consider the following initial value problem (IVP).
$$
\begin{cases}
y' = f(t,y), \quad a\leq t \leq b \\
y(a) = y_a.
\end{cases}
$$
\begin{parts}
\part Give the Taylor series expansion for $y(t+h)$ with center $t$ up til error $O(h^3)$. \vfill
\part Derive Euler's method for the above IVP. What is the local error and global error? \vfill
\part Derive an order 2 Taylor method for the above IVP. What is the local error and global error? \vfill
\part Write down the methods from (b) and (c) for the following IVP.
$$
\begin{cases}
y' = 2ty, \quad 0\leq t \leq 1 \\
y(0) = 1.
\end{cases}
$$ \vfill
\end{parts}


\end{questions}
\end{document} 