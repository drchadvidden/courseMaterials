\documentclass{article}
\usepackage{amsmath}
\usepackage[margin=0.5in]{geometry}
\usepackage{amssymb,amscd,graphicx}
\usepackage{epsfig}
\usepackage{epstopdf}
\usepackage{hyperref}
\usepackage{color}
\usepackage[]{amsmath}
\usepackage{amsfonts}
\usepackage{amsthm}
\bibliographystyle{unsrt}
\usepackage{amssymb}
\usepackage{graphicx}
\usepackage{epsfig}  		% For postscript
%\usepackage{epic,eepic}       % For epic and eepic output from xfig
\renewcommand{\thesection}{}  % toc dispaly

\newcommand{\ds}{\displaystyle}
\newtheorem{thm}{Theorem}[section]
\newtheorem{prop}[thm]{Proposition}
\newtheorem{lem}[thm]{Lemma}
\newtheorem{cor}[thm]{Corollary}
\title{Calculus III Notes}
\date
\Large
\begin{document}
\maketitle
\large

\tableofcontents


%%%%%%%%%%%%%%%%%%%%%%%%%%%%%%%%%%%%%%%
%%%%%%%%%%%%%%%%%%%%%%%%%%%%%%%%%%%%%%%
\section{Fun stuff}

%%%%%%%%%%%%%%%%%%%%%%%%%%%%%%%%%%%%%%%
%%%%%%%%%%%%%%%%%%%%%%%%%%%%%%%%%%%%%%%
\section{Chapter 12 Vectors and the geometry of space}
\subsection{12.1 Three-dimensional coordinate systems}
\begin{enumerate}
\item Rectangular (Cartesian) coordinate system
\begin{enumerate}
%%%%%%%%%%
\item 2D: 
\begin{itemize}
\item Basics: $xy$-plane, orthogonal axis with standard orientation, 4 quadrants, coordinates of point $(x,y)$, projection onto axis, notation $\mathbb{R}^2 = \mathbb{R} \times \mathbb{R} = \{ (x,y) | x,y \in \mathbb{R} \}$.
\item Distance between two points, Pythagoras, distance formula.
\end{itemize} 

%%%%%%%%%%
\item 3D: 
\begin{itemize}
\item $xyz$-space, orthogonal axis with standard orientation, 8 octants, coordinates of point $(x,y,z)$, projection onto $xy$-plane ($xz,yz$). Projection onto axis,  notation $\mathbb{R}^3 = \{ (x,y,z) | x,y,z \in \mathbb{R} \}$.
\item Distance between two points, Pythagoras twice, distance formula, proof in text.
\end{itemize}
\end{enumerate}

\item Graphs of equations
\begin{enumerate}
%%%%%%%%%%
\item 2D: 
\begin{itemize}
\item Point
\item Lines: Vertical, horizontal, sloped
\item Quadratics, polynomials
\item Circles from distance formula, standard form $(x-1)^2+(y-2)^2=4$, complete the square if not in standard form.
\item Lots more
\item Try on own: Regions via inequalities $y<x$, $x^2+y^2>9$, $x/y<1$, $xy\geq0$.
\end{itemize}
%%%%%%%%%%
\item 3D: 
\begin{itemize}
\item Point
\item Planes: Vertical ($x=2$), horizontal ($z=1)$, out at us ($y=0$).
\item Try on own: $x^2+y^2=1$, $x+y=1$, $z=x^2$, $x<y$.
\item Spheres from the distance formula, standard form, complete the square if not in standard form.
\item Showcase Geogebra.
\end{itemize}
\end{enumerate}

%%%%%%%%
\item Homework: 7, 9, 11, 13, 15, 17, 21, 23, 25-37 odd, 45
\end{enumerate}

\subsection{12.2 Vectors}
\begin{enumerate}
%%%%%%%%%%%
\item Vector basics: $\mathbb{R}^2$, then $\mathbb{R}^3$. 
\begin{enumerate}
\item Coordinate (location) vs vector (action such as displacement).
\item Vector has 2 attributes, magnitude (size) and direction (angle).
\item Location doesn't matter, standard position for comparison.
\item Vector components.
\[
\vec{a} = \langle a_1, a_2 \rangle = \langle x, y \rangle
\]
\item Vector from two points $\vec{AB}$. General formula.
\item Magnitude and direction. Need to adjust direction by $180^{\circ}$ with arctangent formula for quadrants 2 and 3.
\[
\| \vec{a} \| = \sqrt{a_1^2+a_2^2}, \quad \theta = \arctan(y/x)
\]
\end{enumerate}

%%%%%%%%%%%
\item Vector operations: Geometry and algebra, $\mathbb{R}^2$ then $\mathbb{R}^3$
\begin{enumerate}
\item Addition: Parallelogram law, sum of components.
\item Scalar multiplication: Stretch / reverse, scale components.
\item Subtraction: Triangular law, subtract components, rewrite as
\[
\vec{a}-\vec{b} = \vec{a}+(-\vec{b}).
\]
\item Bonus: Dot product
\end{enumerate}

%%%%%%%%%%%
\item Theorem: Vector properties, all proven component-wise via properties of real number arithmetic, geometric intuition.
\begin{enumerate}
\item Commutative: $\vec{a} + \vec{b} = \vec{b}+\vec{a}$
\item Associative: $(\vec{a}+\vec{b})+\vec{c} = \vec{a} + (\vec{b}+\vec{c})$
\item Addition identity: $\vec{a}+\vec{0} = \vec{a}$
\item Addition inverse: $\vec{a}+(-\vec{a})=0$
\item Scalar distribution: $c(\vec{a}+\vec{b}) = c\vec{a}+c\vec{b}$
\item Vector distribution: $(c+d)\vec{a} = c\vec{a}+d\vec{a}$
\item Scalar associative: $(cd)\vec{a} = c(d\vec{a})$
\item Scalar multiplication identity: $1\cdot \vec{a} = \vec{a}$
\end{enumerate}

%%%%%%%%%%%
\item Unit vectors and standard basis
\begin{enumerate}
\item $\mathbb{R}^2$: $\langle 1,0 \rangle, \quad \langle 0,1 \rangle$, divide by length to make unit.
\[
\vec{a} = \langle a_1, a_2 \rangle = a_1 \langle 1,0 \rangle + a_2 \langle 0,1 \rangle, \quad \pm \frac{1}{\| \vec{a} \|} \vec{a}
\]
\item $\mathbb{R}^3$: $\vec{i}, \vec{j}, \vec{k}$
\[
\vec{a} = a_1 \vec{i} + a_2 \vec{j} + a_3 \vec{k}
\]
\item $\mathbb{R}^2$: Vector in terms of angle and magnitude.
\[
\vec{a} = \|\vec{a}\| \langle \cos(\theta), \sin(\theta) \|
\]
\end{enumerate}

%%%%%%%%%%%
\item Application: Wire tension. Hang from a wire, wonder if will break. Know angles from ceiling. How much tension on each wire?

%%%%%%%%
\item Homework: 3, 5, 7, 11, 13, 15, 17, 19, 21, 25, 29, 31, 35, 39, 45, 47

\end{enumerate}

\subsection{12.3 The dot product}
\begin{enumerate}
%%%%%%%%
\item Basics of the dot product:
\begin{enumerate}
\item Definition: $\mathbb{R}^2$: $\vec{a} \cdot \vec{b} = a_1b_1 + a_2b_2$,  $\mathbb{R}^3$: $\vec{a} \cdot \vec{b} = a_1b_1 + a_2b_2 + a_3b_3$, 
\item Examples. Note result is a scalar, not a vector.
\item Theorem: Properties of the dot product.
\begin{itemize}
\item $\vec{a} \cdot \vec{a} = \|\vec{a}\|^2$
\item $\vec{a} \cdot \vec{b} = \vec{b} \cdot \vec{a}$
\item $\vec{a} \cdot (\vec{b} + \vec{c}) = \vec{a} \cdot \vec{b} +  \vec{a} \cdot \vec{c})$
\item $(c \vec{a}) \cdot \vec{b} = c(\vec{a} \cdot \vec{b}$
\item $\vec{a} \cdot \vec{0} = 1$
\item All are easily shown via the def of dot product. Show first two quick.
\end{itemize}
\end{enumerate}

%%%%%%%%
\item Meaning of the dot product.
\begin{enumerate}
\item Theorem: For $\theta$ the smallest angle between $\vec{a}$ and $\vec{b}$.
\[
\vec{a} \cdot \vec{b} = \|\vec{a} \| \| \vec{b} \| \cos(\theta)
\]
\item Proof: Law of cosines (generalized Pythagoras, after peek at proofs of LoC) and dot product properties.
\item Why useful? Corollary:
\[
\cos(\theta) = \frac{\vec{a} \cdot \vec{b}}{\|\vec{a}\| \|\vec{b}\|} \quad \rightarrow \quad \theta = \arccos\left(\frac{\vec{a} \cdot \vec{b}}{\|\vec{a}\| \|\vec{b}\|}\right) \in [0,\pi]
\]
\item Corollary: $\vec{a} \perp \vec{b}$ if and only if $\vec{a} \cdot \vec{b}=0$.
\item Example: Find angle between vectors. Show vectors perpendicular. $\vec{0}$ is perpendicular to all vectors. Acute and obtuse cases.
\end{enumerate}

%%%%%%%%
\item Use of dot product, vector orientation.
\begin{enumerate}
\item Direction angles and direction cosines.
\item $\mathbb{R}^3$: Let $\alpha$ be the angle between $\vec{a}$ and $\vec{i}$. Likewise for angles $\beta, \gamma$ and $\vec{j}$ and $\vec{k}$. 
\item $\cos(\alpha) = \frac{\vec{a}\cdot\vec{i}}{\|\vec{a}\| \|\vec{i}\|} = \frac{a_1}{\|\vec{a}\|}$. Likewise for $\cos(\beta)$, $\cos(\gamma)$. 
\item Theorem:
\[
\frac{1}{\| \vec{a} \|} \vec{a} = \langle \cos(\alpha), \cos(\beta), \cos(\gamma) \rangle
\]
\item Example: Find the direction angles of $\vec{a} = \langle 1,2,3 \rangle$.
\end{enumerate}

%%%%%%%%
\item Use of dot product 2, vector projection.
\begin{enumerate}
\item Definitions:
\begin{enumerate}
\item Scalar projection of $\vec{b}$ onto $\vec{a}$: $\text{comp}_{\vec{a}} (\vec{b})$
\item Vector projection of $\vec{b}$ onto $\vec{a}$: $\text{proj}_{\vec{a}} (\vec{b})$
\end{enumerate}
\item Find each using cosine of the angle between and dot product connection to $\cos(\theta)$.
\item Theorem:
\[
\text{comp}_{\vec{a}} (\vec{b}) = \frac{\vec{a} \cdot \vec{b}}{\|\vec{a}\|}, \quad\quad \text{proj}_{\vec{a}} (\vec{b}) = \left(\frac{\vec{a} \cdot \vec{b}}{\|\vec{a}\|} \right) \frac{1}{\|\vec{a}\|}\vec{a} = \frac{\vec{a} \cdot \vec{b}}{\|\vec{a}\|^2} \vec{a}
\]
\item Can see the projection is parallel to $\vec{a}$.
\item Examples
\end{enumerate}

%%%%%%%%
\item Physics application, projection as a way to calculate work.

%%%%%%%%
\item Dot product, cosine similarity, recommender systems. Coding demo.

%%%%%%%%
\item Homework: 1, 3, 7, 9, 13, 15, 19, 23, 27, 29, 33, 39, 43, 45, 47, 61

\end{enumerate}

\subsection{12.4 The cross product}
\begin{enumerate}

%%%%%%%%%%%
\item Basics of the cross product:
\begin{enumerate}
\item Given two non-parallel vectors, find a third non-zero vector which is orthogonal to both. Will use this idea to define planes / tangent planes later on.
\item Given $\vec{a}, \vec{b}$ not parallel, want $\vec{c}$ such that
\[
\vec{a} \cdot \vec{c} = a_1c_1+a_2c_2+a_3c_3=0 \quad \text{and} \quad \vec{b} \cdot \vec{c} = b_1c_1+b_2c_2+b_3c_3=0.
\]
Eliminate $c_3$ by multiplying two equations and subtracting to get
\[
a_1b_3 c_1 + a_2b_3 c_2 - a_3b_1 c_1 - a_3b_2 c_2 = 0
\]
which gives
\[
(a_1b_3 - a_3b_1) c_1 + (a_2b_3 - a_3b_2) c_2 = 0.
\]
Choose $c_1 = (a_2b_3 - a_3b_2)$ and $c_2 = (a_1b_3 - a_3b_1)$ which yields $c_3 = (a_1b_2 - a_2b_1)$. 
\item Definition: The cross product of $\vec{a}$ and $\vec{b}$ is
\[
\vec{a} \times \vec{b} = \langle a_2b_3 - a_3b_2, a_1b_3 - a_3b_1, a_1b_2 - a_2b_1 \rangle.
\]
Note, result is a vector where the dot product gives a scalar.
\item Theorem: $\vec{a} \times \vec{b}$ is orthogonal to both $\vec{a}$ and $\vec{b}$. Proof just computes $(\vec{a} \times \vec{b}) \cdot \vec{a}$. Same for $\vec{b}$. 
\item Determinant notation:
\[
\vec{a} \times \vec{b} = \left|
\begin{array}{ccc}
\vec{i} & \vec{j} & \vec{k} \\
a_1 & a_2 & a_3 \\
b_1 & b_2 & b_3 
\end{array}
\right| 
= \vec{i}
\left|
\begin{array}{cc}
a_2 & a_3 \\
b_2 & b_3
\end{array}
\right| +
\vec{j}
\left|
\begin{array}{cc}
a_1 & a_3 \\
b_1 & b_3
\end{array}
\right| +
\vec{k}
\left|
\begin{array}{cc}
a_1 & a_2 \\
b_1 & b_2
\end{array}
\right|
\]
\item Example: Find the cross product of two random vectors. Check that worked. What if vectors parallel? One zero?
\item Orientation of $\vec{a} \times \vec{b}$ and the right hand rule.
\end{enumerate}

%%%%%%%%%%%
\item Information hidden in the cross product.
\begin{enumerate}
\item Theorem: $\| \vec{a} \times \vec{b} \| = \|\vec{a} \| \|\vec{b} \| \sin(\theta)$. See proof in text. Easy except for first part. Surprising at first, but can see just comes from the dot product result.
\item Corollary: Two nonzero vectors are parallel if and only if $\vec{a} \times \vec{b} = \vec{0}$.
\item Corollary:  $\| \vec{a} \times \vec{b} \|$ gives the area of the parallelogram formed by $\vec{a}$ and $\vec{b}$. Draw parallelogram. Base times height.
\item Find the area of the triangle in $\mathbb{R}^3$ formed by three random points.
\end{enumerate}

%%%%%%%%%%%
\item Properties of the cross product.
\begin{enumerate}
\item Consider combinations of cross product of unit basis $\vec{i}, \vec{j}, \vec{k}$. Note in general $\vec{a} \times \vec{b} \neq \vec{b} \times \vec{a}$ because of right hand rule. Also since orthogonal basis, $\sin(\pi/2)=1$ and can see the result is unit. Parallelogram is a square.
\item Theorem: Properties of the cross product.
\begin{itemize}
\item $\vec{a} \times \vec{b} = - \vec{b} \times \vec{a}$
\item $(c\vec{a}) \times \vec{b} = c(\vec{a} \times \vec{b}) = \vec{a} \times (c\vec{b})$
\item $\vec{a} \times (\vec{b}+\vec{c}) = \vec{a} \times \vec{b}+ \vec{a} \times\vec{c}$
\item $(\vec{a} + \vec{b}) \times \vec{c} = \vec{a} \times \vec{c}+ \vec{b} \times\vec{c}$
\item $\vec{a} \cdot (\vec{b} \times \vec{c}) = (\vec{a} \times \vec{b} ) \cdot \vec{c}$
\item $\vec{a} \times (\vec{b} \times \vec{c}) = (\vec{a} \cdot \vec{c})\vec{b} - (\vec{a} \cdot \vec{b}) \vec{c}$
\item All are proven via the component-wise definition of the cross product.
\end{itemize}
\end{enumerate}

%%%%%%%%%%%
\item Triple product, volume of parallelpiped.
\begin{enumerate}
\item $3 \times 3$ determinant.
\[
\vec{a} \cdot (\vec{b} \times \vec{c}) = \vec{a} \cdot \left|
\begin{array}{ccc}
\vec{i} & \vec{j} & \vec{k} \\
b_1 & b_2 & b_3 \\
c_1 & c_2 & c_3 
\end{array}
\right|
= \left| \begin{array}{ccc}
a_1 & a_2 & a_3 \\
b_1 & b_2 & b_3 \\
c_1 & c_2 & c_3 
\end{array}
\right|
\]
\item Theorem: The volume of the parallel-piped formed by $\vec{a}, \vec{b}, \vec{c}$ is
\[
\| \vec{a} \cdot (\vec{b} \times \vec{c}) \|
= \| \vec{a} \| \| \vec{b} \times \vec{c}) \| \cos(\theta)
\]
where $ \| \vec{b} \times \vec{c}) \|$ is the area of the base and $\| \vec{a} \| \cos(\theta)$ is the height. This comes from our dot product formula.
\item Corollary: $\vec{a}, \vec{b}, \vec{c}$ are coplanar if and only if the triple product is zero.
\item Newton used this to derive Kepler's law of planetary motion. 
\end{enumerate}

%%%%%%%%%%%
\item Torque definition and magnitude.

%%%%%%%%%%%
\item Homework: 1, 5, 7, 9, 11, 13, 17, 19, 23, 25, 27, 31, 33, 37, 49, 53
\end{enumerate}

\subsection{12.5 Equations of lines and planes}
\begin{enumerate}

%%%%%%%%%%%%%%
\item Equations of lines: Vector, parametric, symmetric.
\begin{enumerate}
%%%%%%%%%%%%%%%%%%
\item $\mathbb{R}^2$
\begin{itemize}
\item Familiar case: $y=mx+b$, Ex $y=2x+1$, graph it.
\item Two step process: Get to the line via $\vec{r}_0$, traverse the line via $\vec{v}$ which is parallel to the line.
\item Ex: $\vec{r}_0 = \langle 0,1 \rangle$, $\vec{v} = \langle 1,2 \rangle$, then
\[
\vec{r} = \vec{r}_0 + t\vec{v} = \langle t, 1+2t \rangle.
\]
Since $t=x$, we have $y=2x+1$ again.
\item Parameter $t$ moves us along the line in a direction as $t$ increases. \item Vector form is not unique. $\vec{r}=\vec{r}_0-t\vec{v}$ would give the same line, just traced backwards.
\end{itemize}
%%%%%%%%%%%%%%%%%%
\item $\mathbb{R}^3$
\begin{itemize}
\item Vector equation: For $\vec{v}$ parallel to the line and $\vec{r}_0$ the vector from the origin to any point on the line,
\[
\vec{r} = \vec{r}_0 + t\vec{v} = \langle x_0+at, y_0+bt, z_0+ct \rangle
\]
\item Draw picture.
\item Parametric equations of a line: For parameter $t$,
\[
\begin{cases}
x=x_0+at \\
y=y_0+bt \\
z=z_0+ct.
\end{cases}
\]
PEs are not unique though they may draw the same line. 
\item Symmetric equations of a line: Solve for parameter $t$.
\[
\frac{x-x_0}{a} = \frac{y-y_0}{b} = \frac{z-z_0}{c} 
\]
It is possible that $a,b,c$ could be zero.
\end{itemize}

%%%%%%%%%%%%
\item Example: Find the vector, parametric, and symmetric equations for the line thru two random points. Where does it intersect the $xy$-plane? $xz$? $yz$?

%%%%%%%%%%%%
\item 3 possibilities for lines meeting now: parallel, intersecting, or skew (not parallel, not intersecting).
\begin{itemize}
\item 3 lines, decide if pairs are parallel, intersecting, or skew. Graph in Geogebra.
\end{itemize}

%%%%%%%%%%%
\item Line segment from point $(x_0, y_0, z_0)$ to $(x_1, y_1, z_1)$:
\[
\vec{r} = (1-t) \vec{r}_0 + t\vec{r}_1, \quad \vec{r}_0 = \langle x_0, y_0, z_0 \rangle, \quad \vec{r}_1 = \langle x_1, y_1, z_1 \rangle, \quad 0 \leq t \leq 1.
\]
\end{enumerate}

%%%%%%%%%%%%%%
\item Equations of planes: Vector, scalar, linear
\begin{enumerate}
\item Harder to define the direction of a plane. Normal (perpendicular) vector does the trick.
\item Vector equation of plane: For $(x_0, y_0, z_0)$ a fixed point on the plane, any point $(x,y,z)$ on the plane, and $\vec{n}=\langle a,b,c \rangle$ a normal vector to the plane, we have that
\[
\vec{n} \cdot (\vec{p}-\vec{p_0}) = 0
\]
where $\vec{p_0}=(x_0,y_0,z_0)$ and $\vec{p}=(x,y,z)$. Draw picture to illustrate.
\item Scalar equation of plane:  Compute $\vec{n} \cdot (\vec{p}-\vec{p_0}) = 0$.
\[
a(x-x_0) + b(y-y_0) + c(z-z_0) = 0
\]
\item Linear equation of plane: Combine constant terms of $a(x-x_0) + b(y-y_0) + c(z-z_0) = 0$.
\[
ax+by+cz+d=0
\]
\item Example: Find the plane thru three random points. Uniquely possible if points are not colinear. Already have point, use cross product to get normal vector. Give all 3 forms. Plot the plane by computing the axis intercepts. Check with Geogebra.
\end{enumerate}

%%%%%%%%%%%%%%
\item Summary: In $\mathbb{R}^3$,
\begin{enumerate}
\item You need a point and a direction (parallel vector) to define a line.
\item You need a point and a normal vector to define a plane.
\item Examples: Group challenge.
\begin{itemize}
\item Problems in text: 35, 37, 45, 51.
\end{itemize}
\end{enumerate}

%%%%%%%%%%%%%%
\item Homework: 1, 3, 5, 7, 11, 13, 15, 17, 19, 23, 29, 31, 35, 37, 39, 41, 45, 49, 51, 53, 55, 59, 63, 65 

\end{enumerate}

\subsection{12.6 Cylinders and quadratic surfaces}
\begin{enumerate}

%%%%%%%%%%
\item Summary: Goal is to develop intuition for $\mathbb{R}^3$.
\begin{enumerate}
\item We already considered two classes of surfaces in $\mathbb{R}^3$: Spheres and planes.
\[
(x-a)^2+(y-b)^2+(z-c)^2=r^2, \quad ax+by+cz+d=0
\]
\item New surfaces for this section:
\begin{itemize}
\item Cylinders: Surfaces consisting of all lines (called \emph{rulings}) parallel to a given line and passing thru a planar curve.
\item Example: $z=x^2$ is a parabolic cylinder. Parabolas are called vertical \emph{traces}.
\item Terminology: A \emph{trace} is a curve of intersection of the surface with planes parallel to the coordinate planes ($xy, xz, yz$).
\item Quadratic surface: Any surface generated by the general equation
\[
Ax^2+By^2+Cz^2+Dxy+Exz+Fyz+Gx+Hy+Iz+J=0
\]
\item Example: $x^2+y^2+z^2=1$ is a sphere.
\item Quadratic surfaces: 
\end{itemize}
\end{enumerate}

%%%%%%%%%%
\item Cylinders: Sketch the graph. What are the traces? What are the rulings?
\begin{enumerate}
\item Example: $x^2+z^2=4$
\item Example: $y=z^2$
\end{enumerate}

%%%%%%%%%%
\item Quadratic surfaces: Sketch the traces, then the graph.
\begin{enumerate}
\item Cone: $z^2=x^2+y^2$.
\item Elliptic paraboloid: $z=x^2+y^2$
\item Hyperbolic paraboloid: $z=x^2-y^2$
\item Recall the formula for an ellipse of width $2a$ and height $2b$ centered at the origin. Circle is a special case.
\[
\frac{x^2}{a^2}+\frac{y^2}{b^2}=1
\]
\item Show text table of 6 classes, won't test the hyper-stuff. Just basics.
\end{enumerate}

%%%%%%%%%%
\item Homework: 1,3,5,7,11,17,21,23,25,27

\end{enumerate}


%%%%%%%%%%%%%%%%%%%%%%%%%%%%%%%%%%%%%%%
%%%%%%%%%%%%%%%%%%%%%%%%%%%%%%%%%%%%%%%
\section{Chapter 13 Vector functions}

\subsection{13.1 Vector functions and space curves}
\begin{enumerate}
\item Finally we do calculus, basic case first: $\vec{r}(t)$ is a vector-valued function. For input $t$, result is a vector. 
\begin{enumerate}
\item Need a function per component.
\[
\vec{r}(t) = \langle f(t), g(t), h(t) \rangle
\]
\item Example: Already know lines. Label direction.
\[
\vec{r}(t) = \langle x_0 + at, y_0 + bt, z_0 + ct \rangle
\]
Knowing two points allows to draw the line. Show example.
\item Example: Corkscrew. Label direction.
\[
\vec{r}(t) = \langle \cos(t), sin(t), t \rangle
\]
Helps to graph the projection onto the $xy, xz, yz$ planes.
\item Example: Try on own.
\[
\vec{r}(t) = \langle t, t^2, t^3 \rangle
\]
\item Matching: Text problems 21-26.

\item Find a vector function to describe the curve of intersection of cylinder $x^2+y^2=4$ and surface $z=xy$. 
\end{enumerate}

%%%%%%%%%%%%%%
\item Limits and continuity: Everything component-wise.
\begin{enumerate}
\item If $\vec{r}(t) = \langle f(t), g(t), h(t) \rangle$, then
\[
\lim_{t \rightarrow a} \vec{r}(t) = \langle \lim_{t \rightarrow a} f(t), \lim_{t \rightarrow a} g(t), \lim_{t \rightarrow a} h(t) \rangle.
\]
\item $\vec{r}(t)$ is continuous at $t=a$ if
\[
\lim_{t \rightarrow a} \vec{r}(t) = \vec{r}(a)
\]
\item Example from previous.
\end{enumerate}

%%%%%%%%%%%%%%%%%%%
\item Homework: 1, 3, 5, 7, 9, 11, 15, 17, 27, 29, 31, 43, 49

\end{enumerate}

\subsection{13.2 Derivatives and integrals of vector functions}

\begin{enumerate}

%%%%%%%%%%%%%
\item Derivatives of vector functions:
\begin{enumerate}
\item Definition: For $\vec{r}(t)$ any vector function, define the derivative as
\[
\vec{r}'(t) = \lim_{h \rightarrow 0} \frac{\vec{r}(t+h) - \vec{r}(t)}{h}.
\]
Draw picture in $\mathbb{R}^3$ for some $t$. Result is a tangent vector at $t$. The unit tangent vector 
\[
\vec{T}(t) = \frac{\vec{r}'(t)}{\|\vec{r}'(t) \|}.
\] 
Also have a tangent line using this vector and point. 

\item Theorem:  In $\mathbb{R}^3$ for vector function $\vec{r}(t) = \langle f(t), g(t), h(t) \rangle$, we have
\[
\vec{r}'(t) = \langle f'(t), g'(t), h'(t) \rangle.
\]
Proof moves the difference quotient inside the vector function component-wise.

\item Examples:
\begin{itemize}
\item $\mathbb{R}^2$ case, tangent vector to $\vec{r}(t) = \langle t-2, t^2+1 \rangle$ when $t=2$. Draw picture. Tangent line also.
\item Tangent vector for any line $\vec{r}(t)=\vec{r}_0 + t\vec{v}$ is $\vec{v}$ and the tangent line is the same line. This is the velocity vector for the line as we will see in the next section. Constant change with constant velocity.
\item Find the tangent vector to $\vec{r}(t) = \langle \cos(t), \sin(t), t \rangle$ at $t=2$. Tangent line also. Geogebra. 
\end{itemize}
\end{enumerate}

%%%%%%%%%%%%%
\item Vector function differentiation rules.
\begin{enumerate}
\item Theorem: For $\vec{u}(t)$ and $\vec{v}(t)$ differentiable vector functions,
\begin{itemize}
\item $\ds \frac{d}{dt} \left[ \vec{u}(t)+\vec{v}(t) \right] = \vec{u}'(t)+\vec{u}'(t)$
\item $\ds \frac{d}{dt} \left[ c\vec{u}(t)\right] = c\vec{u}'(t)$
\item $\ds \frac{d}{dt} \left[ f(t) \vec{u}(t) \right] = f'(t)\vec{u}(t)+f(t)\vec{u}'(t)$
\item $\ds \frac{d}{dt} \left[ \vec{u}(t)\cdot \vec{v}(t) \right] = \vec{u}'(t)\cdot\vec{v}(t)+\vec{u}(t)\cdot\vec{v}'(t)$
\item $\ds \frac{d}{dt} \left[ \vec{u}(t)\times \vec{v}(t) \right] = \vec{u}'(t)\times\vec{v}(t)+\vec{u}(t)\times\vec{v}'(t)$
\item $\ds \frac{d}{dt} \left[ \vec{u}(f(t))\right] = \vec{u}'(f(t)) f'(t)$
\end{itemize}
\item Proof of dot product version, component-wise calculation.
\end{enumerate}

%%%%%%%%%%%%%
\item Homework: 1, 3, 7, 13, 17, 21, 25, 43, 45, 47

\end{enumerate}

\subsection{13.3 Arc length and curvature}

SKIP

\subsection{13.4 Motion in space: Velocity and acceleration}

\begin{enumerate}

%%%%%%%%%%%%%%%%%%%
\item Finally, velocity and speed.
\begin{enumerate}
\item Definition: The velocity vector function $\vec{v}(t)$ of position of particle curve $\vec{r}(t)$ is given by 
\[
\vec{v}(t) = \vec{r}'(t) = \lim_{h \rightarrow 0} \frac{\vec{r}(t+h)-\vec{r}(t)}{h}.
\]
Draw picture in $\mathbb{R}^3$. This give speed and direction.
\item Definition: The speed of particle at position $\vec{r}(t)$ is 
\[
\|\vec{v}(t)\| = \|\vec{r}'(t)\| = \frac{ds}{dt}/
\]
\item Definition: The acceleration is
\[
\vec{a}(t) = v'(t).
\]
\item Example: A parameterization of $y=x^2$ is given by $\vec{r}(t) = \langle 2t^2, 4t^4 \rangle$. Plot the velocity and acceleration vectors for $t=1$. Find the speed. Note the direction of the velocity vector is parallel to the old fashion tangent line. 
\end{enumerate}

%%%%%%%%%%%%%%%%%%%
\item Homework: 1, 3, 7, 9, 11, 15, 19, 

%%%%%%%%%%%%%%%%%%%
\item Chapter review problems:
\begin{enumerate}
\item Concept check: 1-4, 8
\item T/F: 1-6, 11, 14
\item Exercises: 1-4, 9, 16-19
\end{enumerate}


\end{enumerate}

%%%%%%%%%%%%%%%%%%%%%%%%%%%%%%%%%%%%%%%
%%%%%%%%%%%%%%%%%%%%%%%%%%%%%%%%%%%%%%%
\section{Chapter 14 Partial derivatives}

Here we return to calculus ideas to extend old idea (functions of one variable $y=f(x)$) to 3 dimensional space (functions of two variables $z=f(x,y)$).
\begin{itemize}
\item 2 dimensions: Get IROC for $f(x)$ as $\frac{df}{dx}$ via AROC as $\frac{\Delta f}{\Delta x}$. Graphs of $y=f(x)$ have tangent lines. Key is idea of limit.
\item 3 dimensions: Functions like $f(x,y)=x^2+y^2$ (and even $f(x,y,z)$) should also have rates of change. Surface analogy. Key will still be limit.
\end{itemize}
Summary of chapter in 6 lines: Curve $y=f(x)$ vs surface $z=f(x,y)$.
\begin{itemize}
\item $\frac{df}{dx}$ becomes two first order derivatives $\frac{df}{dx}$ and $\frac{df}{dy}$
\item $\frac{d^2f}{dx^2}$ becomes four second order derivatives $x^2,xy,yx, y^2$
\item Linear approximation $\Delta f \approx \frac{df}{dx} \Delta x $ becomes $\Delta f \approx  \frac{df}{dx} \Delta x +  \frac{df}{dy} \Delta y$
\item Tangent line $y-y_0 = \frac{df}{dx}(x-x_0)$ becomes a tangent plane $z-z_0 = \frac{df}{dx}(x-x_0)+\frac{df}{dy}(y-y_0)$.
\item Chain rule $\frac{dy}{dt} = \frac{dy}{dx}\frac{dx}{dt}$ becomes $\frac{dz}{dt} = \frac{dz}{dx}\frac{dx}{dt}+\frac{dz}{dy}{dy}{dt}$.
\item Max/min problem $\frac{df}{dx}$ becomes the pair $\frac{df}{dx}, \frac{df}{dy}$.
\end{itemize}

\subsection{14.1 Functions of several variables}
\begin{enumerate}
%%%%%%%%%%%%%%%%%%%%
\item Functions in $\mathbb{R}^2$
\begin{enumerate}
\item $y=f(x)$ is a curve in the $xy$-plane. 
\item $x$ is the indep variable, $y$ is the dependent variable. 
\item Set of all $x$ which $f$ makes sense gives the domain, all obtainable $y$ gives the range. Both are intervals.
\item Example: $f(x) = \sqrt{x}$.
\end{enumerate}

%%%%%%%%%%%%%%%%%%%%
\item Functions in $\mathbb{R}^3$
\begin{enumerate}
%%%%%%%%%%%%%%%%%%%%
\item $z=f(x,y)$ is a surface in $\mathbb{R}^3$. $xy$ are independent and $z$ is dependent. The domain is now a 2 dimensional region, and the range is still an interval. Simple extension, though all these ideas become harder.
%%%%%%%%%%%%%%%
\item Example: $z=f(x,y)=\sqrt{x^2+y^2}$.
\begin{itemize}
\item Need $z\geq 0$ for range. 
\item Level curves: For constant $z=k$ we have circles $k^2 = x^2+y^2$. These are circles, and they grow in diameter as $z$ increases. 
\item Resulting graph is a cone. Check in Geogebra.
\end{itemize}
\end{enumerate}

%%%%%%%%%%%%%%%%%%%%
\item Level curves:
\begin{enumerate}
\item Definition: The level curves of function $f(x,y)$ are the curves with equations $f(x,y)=k$ for constant $k$ in the range of $f$.
\item Example: Find the level curves of $f(x,y)=2x+y$. Level curves are lines $k=2x+y$ which are lines $y=-2x+k$. Graph in $xy$-plane. Result is a plane $z=2x+y$ giving $2x+y-z=0$.
\item Note, different functions (surfaces) can have the same level curves. Compare $f(x,y)=x^2+y^2$ (paraboloid). Different locations though. 
\item Examples: Try on own. Find domain and range. Sketch level curves. Describe surface.
\[
z = \frac{y}{x}, \quad z=\sqrt{4-x^2-y^2}
\]
\item Ideas extend to functions of 3+ variables as you think, harder to visualize.
\[
f(x,y,z), \quad f(x_1, x_2, \dots, x_n)
\]
\end{enumerate}

%%%%%%%%%%%%%%%%%%%%
\item Contour maps and calculus intuition: Show contour map of mountain with rivers.
\begin{enumerate}
\item Contours are drawn every 100 ft increase. What do you see?
\item Steep trails have close curves. Flat are far apart.
\item Creeks run perpendicular to level curves. Steepest direction is perpendicular.
\item Loops indicate peaks and troughs.
\item What if you walk along a level curve? No change in elevation.
\end{enumerate}

%%%%%%%%%%%%%%%%%%%%
\item Homework: 1, 7, 11, 13, 15, 19, 23, 25, 33, 35, 37, 41, 43, 49, 61, 63, 65

\end{enumerate}

\subsection{14.2 Limits and continuity}
\begin{enumerate}
%%%%%%%%%%%%%%
\item Limits in $\mathbb{R}$
\begin{enumerate}
%%%%%%%%%%
\item Intuition definition: $\ds \lim_{x\rightarrow a} f(x) = L$ if for $x$ near $a$, $f(x)$ is near $L$. Draw picture. Idea is clear, but need precision to build a theory on. 
%%%%%%%%%%
\item Precise definition: $\ds \lim_{x\rightarrow a} f(x) = L$ if for any $\epsilon > 0$ (no matter how near to $L$), there exists a $\delta >0$ (near enough to $a$) such that if $|x-a| < \delta$, then $|f(x)-L| < \epsilon$. Add $\delta$ and $\epsilon$ to graph. $x$ window and $y$ window. Technical definition which is hard to work with, instead prove theorems and build theory.
%%%%%%%%%%%%%%
\item Techniques for computing limits:
\begin{itemize}
\item Limit laws (solid foundation, grow complexity from basic functions).
\item Algebra tricks (multiply by conjugate, right / left limits, etc).
\item Squeeze theorem and indirect attacks.
\item Can direct substitute for continuous functions.
\end{itemize}
%%%%%%%%%%%%%%%%
\item Why are limits important? Handling indeterminate form. Essence of calculus.
\[
f'(x) = \lim_{h \rightarrow 0} \frac{f(x+h)-f(x)}{h}, \quad \int_a^b f(x)~dx = \lim_{n\rightarrow \infty} \sum_{i=0}^n f(x_i^*)\Delta x
\]
$0/0$ and $\infty \cdot 0$ indeterminate forms.
%%%%%%%%%%%%%%%%%
\item Examples: $f(x)=x^2, f'(3)=?$, $\lim_{x \rightarrow 0} \frac{|x|}{x}$, $\lim_{x \rightarrow 2} \frac{\sqrt{x+2}-2}{x-2}$.
\end{enumerate}

%%%%%%%%%%%%%%
\item Limits in $\mathbb{R}^2$ and beyond
\begin{enumerate}
%%%%%%%%%%%%%%%%%%%
\item  Intuition definition: $\ds \lim_{(x,y)\rightarrow (a,b)} f(x,y) = L$ if for $(x,y)$ near $(a,b)$, $f(x,y)$ is near $L$. Draw picture. Now we approach a point $(a,b)$ from all directions, not just right/left. Precision again is needed.
%%%%%%%%%%%%%%%%%
\item Precise definition: $\ds \lim_{(x,y)\rightarrow (a,b)} f(x,y) = L$ if for any $\epsilon > 0$ (no matter how near to $L$), there exists a $\delta >0$ (near enough to $(a,b)$) such that if $\sqrt{(x-a)^2+(y-b)^2} < \delta$, then $|f(x,y)-L| < \epsilon$. Note the appearance of the distance formula, circle with center $(a,b)$. Again this definition is not practical.
%%%%%%%%%%%%%%%%%%%%%
\item Techniques for computing limits:
\begin{itemize}
\item Limit laws from 1 dim generalize, but cannot separate $x$ from $y$.
\item Squeeze theorem and indirect attacks.
\item Can direct substitute for continuous functions (polynomials, rationals in domain, etc).
\item Interesting case again will be indeterminate forms (next section for partial derivatives).
\end{itemize}
%%%%%%%%%%%%%%%%%%%
\item Same idea for 3+ dimensions.
\end{enumerate}

%%%%%%%%%%%%%%
\item Examples:
\begin{enumerate}
%%%%%%%%%%%%%%%%
\item Table example in text. Hint how to explain a limit does not exist. Graph each in Geogebra.
\item Show $f(x,y)=\frac{x^2-y^2}{x^2+y^2}$ has no limit at $(0,0)$ by following paths $x=0$ and $y=0$ and getting different values. Similar to right left limits in $\mathbb{R}$. Graph in Geogebra.
\item Try on own: Show $f(x,y)=\frac{xy}{x^2+y^2}$ has no limit at $(0,0)$ by choosing two paths with different results. Graph in Geogebra.
\item Theorem: If $f \rightarrow L_1$ as $(x,y)\rightarrow (a,b)$ along path $C_1$ and $f \rightarrow L_2$ as $(x,y)\rightarrow (a,b)$ along path $C_2$ with $L_1 \neq L_2$, then $\lim_{(x,y)\rightarrow (a,b)} f(x,y)$ does not exist.
\item Show $\lim_{(x,y)\rightarrow (0,0)} \frac{3x^2y}{x^2+y^2}=0$ via the Squeeze theorem. Key step:
\[
0 \leq \frac{3x^2|y|}{x^2+y^2} = 3|y|\frac{x^2}{x^2+y^2} \leq 3|y|\cdot 1
\]
Can also do from definition. See text.
\item If point is in domain and function is continuous, can do direct substitution.  $\lim_{(x,y)\rightarrow (1,1)} \frac{3x^2y}{x^2+y^2}=0$.
\end{enumerate}

%%%%%%%%%%%%%%
\item Homework: 5, 9, 13, 17

\end{enumerate}

\subsection{14.3 Partial derivatives}
\begin{enumerate}
%%%%%%%%%%%%%
\item One dimension review, $\mathbb{R}$:
\begin{enumerate}
\item For $f(x)$, change in $x$ results in change in $f$. Then average rate of change $\Delta f/\Delta x$ tends to instantaneous rate of change $df/dx$ as $\Delta x \rightarrow 0$. That is,
\[
\frac{df}{dx} = \lim_{\Delta x \rightarrow 0} \frac{\Delta f}{\Delta x} = \lim_{h \rightarrow 0} \frac{f(x+h)-f(x)}{h}.
\]
\item Limits are foundation, but we built a theory of differentiation.
\[
cf(x), f(x)+g(x), f(x)g(x), f(x)/g(x), f(g(x))
\]
and also special functions such as logs, exponentials, trig, etc.
\end{enumerate}

%%%%%%%%%%%%%
\item Two dimensions, $\mathbb{R}^2$: $f(x,y)$
\begin{enumerate}
\item Analogy tangent plane to a surface. Strategy is to allow one variable to change at a time. If $x$ can change for $f(x,y)=x-yx$, then $\Delta f=\Delta x-y\Delta x$ and $\Delta f/\Delta x=1-y$. That is the $x$ derivative of $f(x,y)$ is $1-y$. Hold $y$ constant and differentiate $f$ in $x$. Knowing both will lead to tangent planes (next section).
\item Definition: The partial derivative of $f(x,y)$ with respect to $x$ is 
\[
f_x(x,y) = \frac{\partial f}{\partial x} = \lim_{h \rightarrow 0} \frac{f(x+h,y) - f(x,y)}{h}
\] Similar for $f_y$.
\item Notation: For $f=f(x,y)$,
\[
f_x = f_x(x,y) = \frac{\partial f}{\partial x} = \frac{\partial}{\partial x} f = D_x f
\]
\item All our old differentiation rules hold since $y$ is a constant.
\item Example: $f(x,y)=4-x^2-3y^2$. 
\begin{itemize}
\item Compute $f_x(1,2), f_x, f_y(1,2), f_y$. 
\item Graph via Geogebra to get intuition around $f_x, f_y$. Note if we know $f_x(1,2), f_y(1,2)$, we can get a tangent plane (next section).
\item Note local max at $(0,0)$.
\item Extend to four cases of second derivatives.
\end{itemize}
\item Example: $f(x,y)=x^3+x^2y^3-2y^2$
\begin{itemize}
\item Try on own, all first and second order partials.
\item Compare graph to $f_x$ and $f_y$.
\end{itemize} 
\item Theorem: $f_{xy}=f_{yx}$, order of differentiation doesn't matter. Proof via the MVT.
\item Example: Problem 9 in text.
\end{enumerate}

%%%%%%%%%%%%%
\item Partial differential equations tour:
\begin{itemize}
\item \url{https://en.wikipedia.org/wiki/Partial_differential_equation}
\item \url{https://web.stanford.edu/class/math220b/handouts/heateqn.pdf}
\end{itemize}
%%%%%%%%%%%%%

\item Homework: 5, 7, 9, 11, 13, 15, 21, 25, 33, 45, 51, 53, 61, 63, 81, 97

\end{enumerate}

\subsection{14.4 Tangent planes and linear approximations}

\begin{enumerate}
%%%%%%%%%%%%%%%
\item Recall: $y=f(x)$ version.
\begin{enumerate}
\item The tangent line to $y=f(x)$ at point $(x_0, y_0)$ is
\[
y-y_0 = f'(x_0)(x-x_0) \quad \rightarrow \quad y=L(x)=f'(x_0)(x-x_0)+y_0.
\]
Give example for $f(x)=x^2$ at $x=3$.
\item Linearization approximates $f(x)$ by this line.
\[
y=f(x) \approx L(x)=f'(x_0)(x-x_0)+y_0.
\]
The closer to the tangent point, the better the approximation. Give example.
\item Taylor series and Taylor's theorem continues this vein.
\[
f(x) = f(x_0) + f'(x_0)(x-x_0) + \dots
\]
\end{enumerate}

%%%%%%%%%%%%%%%
\item Extension to $z=f(x,y)$, tangent planes.
\begin{enumerate}
\item Partial derivatives $f_x, f_y$ give the slope of the tangent line to $z=f(x,y)$ in the $x, y$ directions. Draw picture. How to use this to find the tangent line thru a point $(x_0, y_0, z_0)$? Need a point and a normal vector. 
\item Normal vector construction: Find vectors in direction of partial derivative lines.
\begin{itemize}
\item $f_x:$, $y$ held constant, if $x$ increases 1 unit, $z$ increases $f_x$ units. Then, $\vec{a}=\langle 1,0,f_x \rangle$ is parallel to our line.
\item $f_y:$, likewise $\vec{b}=\langle 0,1,f_y \rangle$ works.
\item The normal vector to the tangent plane is then
\[
\vec{n} = \vec{a} \times \vec{b} = \langle -f_x, -f_y, 1\rangle
\]
\end{itemize}
\item Vector form of tangent plane:
\[
\vec{n} \cdot (\vec{p}-\vec{p_0}) = 0 \quad \rightarrow \quad -f_x(x-x_0)-f_y(y-y_0)+(z-z_0)=0
\]
gives
\[
z-z_0=f_x(x-x_0)+f_y(y-y_0)
\]
Note the similarity to the tangent line for $y=f(x)$.
\item Example: Find the tangent line to the paraboloid $z=14-x^2-y^2$ at $(x_0,y_0,z_0)=(1,2,9)$ Graph in geogebra. Both $x,y$ tangent lines are on this plane. All tangent lines for all surface curves as well.
\item Try on own: Find the tangent plane to the sphere $x^2+y^2+z^2=14$ at $(1,2,3)$. Can solve for $z$ taking the positive root or use implicit differentiation with respect to $x,y$. Note the normal vector is in the same direction as the sphere radius when directed to our point.
\item Linearization of $z=f(x,y)$ by the tangent plane.
\[
f(x,y) \approx L(x,y) = f(x_0, y_0)+f_x(x-x_0)+f_y(y-y_0)
\]
Two dimensional Taylor series approximation. Can guess the extension to 3+ independent variables. 
\end{enumerate}

%%%%%%%%%%%%%%%
\item Differentiability of $f(x,y)$:
\begin{enumerate}
\item Remind of differentiability in $\mathbb{R}^2$. Derivative exists. Differentiable implies continuous. 
\item Def: We say $f(x,y)$ is differentiable at point $(a,b)$ if 
\[
\Delta z = f_x(a,b)\Delta x + f_y(a,b) \Delta y +\epsilon_1 \Delta x + \epsilon_2 \Delta y
\]
where $\epsilon_1,\epsilon_2 \rightarrow 0$ as $\Delta x,\Delta y \rightarrow 0$. Basically says can approximate $f$ well by the tangent line.
\item Theorem: If the partial derivatives $f_x,f_y$ exist near $(a,b)$ and are continuous at $(a,b)$, then $f$ is differentiable at $(a,b)$.
\end{enumerate}

%%%%%%%%%%%%%%%
\item Homework: 1, 3, 5, 11, 13, 19, 21

\end{enumerate}

\subsection{14.5 The chain rule}

\begin{enumerate}
%%%%%%%%%%%%%%%%%%%%%
\item 1 dimension: $\frac{d}{dt} f(g(t)$.
\begin{enumerate}
\item Goal is to differentiate function composition. Nested functions are common. Do $g$ first, then $f$ takes it from there.
\[
\frac{d}{dt} f(x(t)) = f'(x(t)) x'(t)
\]
\item Compact notation: $y=f(x)$
\[
\frac{dy}{dt} = \frac{dy}{dx} \frac{dx}{dt}
\]
Right hand side collapses back if canceling were allowed.
\item The chain rule applied to integration is the substitution rule.
\end{enumerate}

%%%%%%%%%%%%%%%%%%%%%
\item 2 dimensions, basic case: $\frac{d}{dt} f(x(t), y(t))$
\begin{enumerate}
\item Extend the dimension 1 case of the chain rule to get for $z=f(x,y)$:
\[
\frac{dz}{dt} = \frac{d}{dt} f(x,y) = \frac{\partial f}{\partial x} \frac{dx}{dt} + \frac{\partial f}{\partial y} \frac{dy}{dt}
\]
Note the similarity to the 1 dimension case.
\item Example: For $z=3xy^2$, $x=\cos(t), y=\sin(t)$, compute $\frac{dz}{dt}$. Check by rewriting $x,y$ in original. Graph in Geogebra, not traveling about the unit circle in $xy$. Consider $t=0, \frac{\pi}{2}$. Rate of change along curve $(x(t),y(t))$.
\end{enumerate}

%%%%%%%%%%%%%%%%%%%%%
\item 2 dimensions, standard case: $\frac{d}{dt} f(x(s,t), y(s,t))$
\begin{enumerate}
\item Repeat the above formula twice.
\[
\frac{dz}{ds} = \frac{d}{ds} f(x,y) = \frac{\partial f}{\partial x} \frac{dx}{ds} + \frac{\partial f}{\partial y} \frac{dy}{ds}
\]
\[
\frac{dz}{dt} = \frac{d}{dt} f(x,y) = \frac{\partial f}{\partial x} \frac{dx}{dt} + \frac{\partial f}{\partial y} \frac{dy}{dt}
\]
\item Example: For $z=3xy^2$, $x=r\cos(\theta), y=r\sin(\theta)$, compute $\frac{dz}{dr}$. Try on own $\frac{dz}{d\theta}, \frac{d^2z}{dr^2}$
\item Second derivatives and converting to polar coordinates. $z=f(x,y)$, $x=r\cos(\theta), y=r\sin(\theta)$
\begin{itemize}
\item Compute $f_{rr}, f_{\theta \theta}$.
\item Turns out $f_{xx}+f_{yy}=f_{rr}+\frac{1}{r}f_r + \frac{1}{r^2}f_{\theta}$
\item This is the polar version of Laplace's equation.
\end{itemize}
\end{enumerate}

%%%%%%%%%%%%%%%%%%%%%
\item Generalizes to any dimension. Show text formula. Work example 5.

%%%%%%%%%%%%%%%%%%%%%
\item Homework: 1, 3, 5, 7, 11, 13, 17, 21, 45, 49

\end{enumerate}

\subsection{14.6 Directional derivatives and gradient vectors}
\begin{enumerate}

%%%%%%%%%%%%%%%%%
\item Directional derivatives: So far we calculate change for $f(x,y)$ in the $x$ direction ($f_x$) or the $y$ direction ($f_y$), but of course $f$ can change in any direction.
\begin{enumerate}
%%%%%%%%%%%%%%
\item Recall our limit definitions:
\[
f_x(x,y) = \lim_{h \rightarrow 0} \frac{f(x+h,y)-f(x,y)}{h}, \quad
f_y(x,y) = \lim_{h \rightarrow 0} \frac{f(x,y+h)-f(x,y)}{h}
\]
We essentially hold $y$ and $x$ constant respectively. The directions we consider here are $\vec{i}$ and $\vec{j}$. Note both are unit vectors.
%%%%%%%%%%%%%%
\item Example: Find the change in $f(x,y)=xy$ at point $(3,1)$ in the direction $\vec{v}=\langle 1,2 \rangle$. Normalize our direction via the unit vector $\vec{u} = \langle 1/\sqrt{5}, 2/\sqrt{5} \rangle$. Then our change is from $(3,1)$ to $(3+h/\sqrt{5}, 2h/\sqrt{5})$ and
\[
D_{\vec{u}} f(3,1) = \lim_{h \rightarrow 0} \frac{f(3+h/\sqrt{5}, 1+2h/\sqrt{5})-f(3,1)}{h} =  \lim_{h \rightarrow 0} 7/\sqrt{5}+2h/5 = 7\sqrt{5}.
\]
Note $h$ in the denominator because of the unit vector. Graph in Geogebra and compare to $f_x, f_y$.
%%%%%%%%%%%%%%
\item Definition: The directional derivative of $f(x,y)$ at point $(x_0,y_0)$ in the direction of unit vector $\vec{u}=\langle a,b \rangle$ is 
\[
D_{\vec{u}} f(x_0, y_0) = \lim_{h \rightarrow 0} \frac{f(x_0+ah, y_0+bh)-f(x_0,y_0)}{h} 
\]
Note, $D_{\vec{i}} f = f_x$ and $D_{\vec{j}} f = f_y$. Also any unit vector can be expressed in terms of a direction angle $\theta$ as
\[
\vec{u} = \langle a,b \rangle = \langle \cos(\theta), \sin(\theta) \rangle
\]
\end{enumerate}

%%%%%%%%%%%%%%%%%
\item Computing directional derivatives
\begin{enumerate}
%%%%%%%%%
\item The above limit definition is messy to compute. Instead, we rewrite $D_{\vec{u}} f$ in terms of $f_x$ and $f_y$. This seems doable considering the tangent plane to a surface in $\mathbb{R}^3$.
%%%%%%%%%
\item Theorem: For $f(x,y)$ differentiable in both $x$ and $y$ and $\vec{u} = \langle a,b \rangle$ any unit vector in $\mathbb{R}^2$, 
\[
D_{\vec{u}} f(x,y) = f_x(x,y) a + f_y(x,y) b.
\]
%%%%%%%%%
\item Proof: Define $g(h) = f(x_0+ah, y_0+bh)$. Then,
\[
g'(0) = \lim_{h \rightarrow 0} \frac{g(h)-g(0)}{h}  = \lim_{h \rightarrow 0} \frac{f(x_0+ah, y_0+bh)-f(x_0,y_0)}{h} =  D_{\vec{u}} f(x_0, y_0).
\]
On the other hand, from the chain rule,
\[
g'(h) = \frac{\partial f}{\partial h} = \frac{\partial f}{\partial x}\frac{\partial x}{\partial h} + \frac{\partial f}{\partial y}\frac{\partial y}{\partial h} = af_x+bf_y= af_x(x_0+ah,y_0+bh)+bf_y(x_0+ah,y_0+bh).
\]
Evaluating $g'(h)$ at zero and comparing to before gives the result.
%%%%%%%%%
\item Example: Repeat above example $f(x,y)=xy$ with new calculation.
%%%%%%%%%
\item Example: Try on own for $f(x,y)=xy^3-x^2$ at $(1,2)$ in direction $\theta = \frac{\pi}{3}$. Check via Geogebra.
\end{enumerate}

%%%%%%%%%%%%%%%%%
\item Gradient vectors:
\begin{enumerate}
%%%%%%%%%
\item Example: Hint to bigger things. $f(x,y)=3x+y+1$ at $(1,1)$.
\begin{itemize}
\item $\vec{i}$ and $\vec{j}$ directions.
\item No change (level curve) direction. Find $\vec{u} = \langle a,b \rangle$ such that 
\[
D_{\vec{u}} f = f_x a + f_y b = 0
\]
gives $\vec{u} = \frac{1}{\sqrt{10}}\langle -3,1 \rangle$,
\item Perpendicular to level curve gives steepest direction $\vec{u} = \frac{1}{\sqrt{10}}\langle 1, 3 \rangle$. This matches $\langle f_x, f_y \rangle$ at our point. Compute change and compare to $f_x, f_y$.
\item Noting that the directional derivative is really a dot product, we see a new vector of import.
\[
D_{\vec{u}} f = f_x a + f_y b  = \langle f_x, f_y \rangle \cdot \langle a,b \rangle
\] 
\end{itemize}
\item Definition: For $f(x,y)$, the gradient of $f$ is a vector-function of the form
\[
\nabla f = \langle f_x, f_y \rangle
\]
\item Example: Compute gradient for previous example $f(x,y)=xy^3-x^2$ at $(1,2)$. Reproduce previous result.
\item Theorem: First importance of the gradient. For $f$ differentiable, the maximum value of the directional derivative $D_{\vec{u}} f$ is $|\nabla f|$ and is in the direction of $\nabla f$.
\item Proof: We use the law of cosines version of the dot product.
\[
\vec{a} \cdot \vec{b} = \|\vec{a}\| \|\vec{b}\| \cos(\theta)
\]
where $\theta$ is the angle between $\vec{a}, \vec{b}$. Then,
\[
D_{\vec{u}} f = \nabla f \cdot \vec{u} = \|\nabla f\| \|\vec{u}\| \cos(\theta) = \|\nabla f\| \cos(\theta) \leq \|\nabla f\|
\]
which occurs when $\theta=0$ meaning $\vec{u}$ and $\nabla f$ are in the same direction.
\item Example: Apply previous theorem to  $f(x,y)=3x+y+1$ at $(1,1)$,  $f(x,y)=xy^3-x^2$ at $(1,2)$. 
\item Example: Try on own. Number 22 in text. Graph in Geogebra.
\end{enumerate}

%%%%%%%%%%%%%%%%%
\item Extension to functions of three variables: $f(x,y,z)$.
\begin{enumerate}
\item Could be in $\mathbb{R}^4$ in which case cannot visualize. Could be an implicit curve $f(x,y,z)$ in $\mathbb{R}^3$.
\item Definition of directional derivative in direction of unit vector $\vec{u}$.
\[
D_{\vec{u}} = \lim_{h \rightarrow 0} \frac{f(x_0+ah,y_0+bh,z_0+ch)-f(x_0,y_0,z_0)}{h}
\]
\item Compute $D_{\vec{u}}$ in terms of partial derivatives.
\[
D_{\vec{u}} = f_x(x,y,z)a + f_y(x,y,z)b + f_z(x,y,z)c = \nabla f \cdot \vec{u}
\]
\item Gradient of $f$ is
\[
\nabla f = \langle f_x, f_y, f_z \rangle.
\]
\item Examples are pretty well the same.
\end{enumerate}

%%%%%%%%%%%%%%%%%
\item Tangent planes to level surfaces
\begin{enumerate}
\item We already have tangent planes to surfaces of the form $z=f(x,y)$ at point $(x_0, y_0, z_0)$.
\[
z-z_0 = (f_x)_0 (x-x_0) + (f_y)_0(y-y_0)
\]
\item This extends implicitly to a level surface $F(x,y,z)=k$ at point $(x_0, y_0, z_0)$.
\[
(F_x)_0 (x-x_0) + (F_y)_0(y-y_0) + (F_z)_0(z-z_0) = 0
\]
Note, the gradient vector $\nabla F$ is our normal vector to the plane (and surface).
\item The normal line then has symmetric equations
\[
\frac{x-x_0}{(F_x)_0} = \frac{y-y_0}{(F_y)_0} = \frac{z-z_0}{(F_z)_0}.
\]
\item Example: Find the tangent plane to the ellipsoid $x^2/4 + y^2 + z^2/9 = 3$ at point $(-2,1,-3)$. Check result in Geogebra.
\end{enumerate}

%%%%%%%%%%%%%%%%%%%%%%%
\item Summary of gradient vector: This section is rich. Summarize the key ideas.
\begin{enumerate}
\item For $f(x,y)$ (or $f(x,y,z)$), $\nabla f$ gives the direction of fastest increase of $f$.
\item $\|\nabla f\|$ is the fastest increase rate (slope).
\item $\nabla f$ is orthogonal to the level curve (or surface).
\end{enumerate}

%%%%%%%%%%%%%%%%%
\item Homework: 5, 7, 9, 11, 15, 19, 23, 25, 27, 29, 37, 39, 41, 49

\end{enumerate}

%%%%%%%%%%%%%%%%%%%%%%%%%%%%%%%%%%%%%%%
%%%%%%%%%%%%%%%%%%%%%%%%%%%%%%%%%%%%%%%
\subsection{14.7 Maximum and minimum values}

\begin{enumerate}

%%%%%%%%%%%%%%%%%%%%%%%%%%%%%%%%%%%%%%%
\item Recall functions of one variable...
\begin{enumerate}
\item Draw $f(x)$ with make an min values. Smooth and continuous on $\mathbb{R}$.
\item $f'(x)=0$ (stationary points) gives locations of horizontal tangents. $f''(x)=0$ discerns the three cases.
\begin{itemize}
\item $f''(x)>0$, local min
\item $f''(x)<0$, local max
\item $f''(x)=0$, inflection point
\end{itemize}
\item Two other cases for extrema: Singular points, end points.
\item Absolute max and mins are ensured by the EVT: Continuous function $f(x)$ on closed interval $[a,b]$ must have a local max and local min.
\end{enumerate}

%%%%%%%%%%%%%%%%%%%%%%%%%%%%%%%%%%%%%%%
\item Definitions for $f(x,y)$.
\begin{enumerate}
\item Local min at $(a,b)$ with local min value $f(a,b)$. Likewise for max.
\item Global min and max.
\end{enumerate}

%%%%%%%%%%%%%%%%%%%%%%%%%%%%%%%%%%%%%%%
\item Extending calulus 1 results:
\begin{enumerate}
\item Theorem: If $f(x,y)$ has a local max or min at $(a,b)$ and $f_x, f_y$ both exist at $(a,b)$, then $f_x(a,b)=0$ and $f_y(a,b)=0$ ($\nabla f = \vec{0}$).
\item If $\nabla f = \vec{0}$ at $(a,b)$, then this is called a stationary point. Not all critical points are local mins or maxes.
\item Example: Find the stationary points of $f(x,y)=3x-x^3-2y^2+y^4$.
\end{enumerate}


%%%%%%%%%%%%%%%%%%%%%%%%%%%%%%%%%%%%%%%
\item How to classify stationary points? Concavity is key, but we need to look in all directions.
\begin{enumerate}
\item 2 examples: $x^2+xy+y^2$ and $x^2+10xy+y^2$. Only $(0,0)$ stationary point for both. Both have two positive partials second $(f_{xx}=f_{yy}=2>0$). Graph in Geogebra to see different behavior.
\item To classify, consider all the second directional derivatives at once. For $f(x,y)$ and $\vec{u} = \langle h,k \rangle$,
\[
D_{\vec{u}} f = f_x h + f_yk.
\]
\[
D_{\vec{u}}^2 f = D_{\vec{u}}\left( f_x h + f_yk \right) = f_{xx} h^2 + 2f_{xy}hk + f_{yy}k^2 = f_{xx} \left(h + \frac{f_{xy}}{f_{xx}} k\right)^2 + \frac{k^2}{f_{xx}} \left(f_{xx}f_{yy}-f_{xy}^2 \right)
\]
where the last step follows by completing the square.
\item If we think concave up since $f_{xx}>0$ we would also need $D=f_{xx}f_{yy}-f_{xy}^2>0$. Likewise for concave down we need $f_{xx}<0$ but still $D>0$.
\item Theorem: For $(a,b)$ a stationary point of $f(x,y)$ and 
\[
D = D(a,b) = f_{xx}f_{yy}-f_{xy}^2 = \left|
\begin{array}{cc}
f_{xx} & f_{xy} \\
f_{yx} & f_{yy} 
\end{array}
\right|
\]
\begin{itemize}
\item If $D>0$ and $f_{xx}(a,b)>0$, then $f(a,b)$ is a local min.
\item If $D>0$ and $f_{xx}(a,b)<0$, then $f(a,b)$ is a local max.
\item If $D>0$ then $f(a,b)$ is a saddle point.
\end{itemize}
\item Check for 2 examples.
\item Example: Apply to first example.
\end{enumerate}

%%%%%%%%%%%%%%%%%%%%%%%%%%%%%%%%%%%%%%%
\item Last, we extend the EVT
\begin{enumerate}
\item EVT: For $f(x,y)$ continuous on closed, bounded region $R$ in $\mathbb{R}^2$, then $f$ has an absolute max and min in $R$.
\item How to find extrema? Abs max and mins must be at stationary points in $R$ or on the boundary of $R$.
\begin{itemize}
\item Find the stationary points in $R$.
\item Find the extreme values on the boundary via Calc 1.
\item Get the largest and smallest $f$ values from parts 1 and 2.
\end{itemize}
\item Example: 34 in text.
\end{enumerate}

%%%%%%%%%%%%%%%%%%%%%%%%%%%%%%%%%%%%%%%
\item Homework: 1, 3, 5, 7, 13, 15, 17, 23, 27, 31, 33

\end{enumerate}


\subsection{14.8 Lagrange multipliers}
Skip.

\subsection{Chapter 14 Review}
\begin{enumerate}
\item Concept check: 1-18
\item True-False: 1-11
\item Exercises: 1-56
\end{enumerate}

%%%%%%%%%%%%%%%%%%%%%%%%%%%%%%%%%%%%%%%
%%%%%%%%%%%%%%%%%%%%%%%%%%%%%%%%%%%%%%%
\section{Chapter 15 Multiple integrals}

\subsection{15.1 Double integrals over rectangles}
\begin{enumerate}

%%%%%%%%%%%%%%%%%%%%
\item Summary of past: Extend the definite integral of calculus 1 to 3 dimensions. 
\begin{enumerate}
\item $\ds \int_a^b f(x) ~dx$ as area under the curve.
\item Compute via limit of Riemann sum. Classic calculus paradox.
\item Fundamental theorem of calculus.
\item Alternate view: $\ds \int_a^b f(x) ~dx$ as adding up 1D lengths to get 2D area.
\item Really about summation: Sum lines to area, areas to volumes (discs and washers), probability, force to work, line segment to arc length, arc length to surface area, etc
\end{enumerate}

%%%%%%%%%%%%%%%%%%%%
\item Basic case for $z=f(x,y)$ in $\mathbb{R}^3$: Volumes over rectangular domains.
\begin{enumerate}
\item Find the volume of the solid between $z=f(x,y)$ and the $xy-$plane over region $R = [a,b] \times [c,d]$ (Cartesian product).
\item Partition $R$ by $\Delta x$ and $\Delta y$ giving rectangular areas $\Delta A$.
\item Notation and limit of a Riemann sum. 
\[
\iint\limits_R f(x) ~dA = \int_a^b \int_c^d f(x,y) ~dy ~dx = \lim_{m,n \rightarrow \infty} \sum_{i=1}^n \sum_{j=1}^n f(x_{ij}^*, y_{ij}^*) \Delta A 
\]
\item If the limit exists we say $f$ is integrable over $R$.
\item Can approximate area as in Calc 1 by computing the finite sum, though only works for simple functions.
\end{enumerate}

%%%%%%%%%%%%%%%%%%%%
\item FTOC for calculation: Volume by accumulating area.
\begin{enumerate}
\item Slicing the solid in the $x$ direction gives cross-sections with area
\[
A(x) = \int_c^d f(x,y) ~dy.
\]
This is a computable function of $x$ for any $y$ held constant. 
\item Add up area to get volume. FTOC twice.
\[
V = \int_a^b A(x) ~dx = \int_a^b \left[ \int_c^d f(x,y) ~dy \right] ~dx
\]
\item Of course slicing in $y$ gives a similar formula resulting in Fubini's theorem which extends to more general regions as well.
\[
\int_a^b \left[ \int_c^d f(x,y) ~dy \right] ~dx = \int_c^d \left[ \int_a^b f(x,y) ~dx \right] ~dy
\]
\item Examples: Divide and conquer to find the area under $z=1+x^2+y^2$ on $[1,2]\times[0,1]$.
\end{enumerate}

%%%%%%%%%%%%%%%%%%%%
\item Average value of functions:
\begin{enumerate}
\item Calc 1 version:
\[
\frac{1}{b-a} \int_a^b f(x) ~dx
\]
\item Calc 3 version:
\[
\frac{1}{A(R)} \iint\limits_R f(x) ~dA
\]
\end{enumerate}

%%%%%%%%%%%%%%%%%%%%
\item Homework: 1, 13, 15, 17, 19, 21, 25, 29, 31, 33, 35, 39, 47

\end{enumerate}


%%%%%%%%%%%%%%%%%%%%
%%%%%%%%%%%%%%%%%%%%
%%%%%%%%%%%%%%%%%%%%
\subsection{15.2 Double integrals over general regions}

\begin{enumerate}

%%%%%%%%%%%%%%%%%%%%
\item Idea of general regions:
\begin{enumerate}
\item The domain of integration for $z=f(x,y)$ doesn't have to be a rectangle. In general it can be any shape (denote as $D$). Draw picture.
\item Can still do slicing in $x$ or $y$ direction. Result is 2 cases to choose from. 
\item Volume from accumulating area in $x$ (holding $y$ constant). Draw domain picture.
\[
\iint\limits_D f(x,y) ~dA = \int_a^b A(x) ~dx = \int_a^b \int_{g_1(y)}^{g_2(y)} f(x,y) ~dy ~dx
\]
\item Volume from accumulating area in $y$ (holding $x$ constant). Draw domain picture.
\[
\iint\limits_D f(x,y) ~dA = \int_c^d A(y) ~dy = \int_c^d \int_{h_1(x)}^{h_2(x)} f(x,y) ~dx ~dy
\]
\item Sometimes only one is an option, sometimes both can be used and need to think strategically.
\end{enumerate}

%%%%%%%%%%%%%%%%%%%%
\item Examples: Drawing the domain in the $xy-$plane is key.
\begin{enumerate}
\item Only one direction is easy. Find the volume under the surface $z=x^2+y$ on domain $D$ bound by curves $y=x+1$ and $y=x^2$.
\item Divide and conquer by doing both at same time. Find the volume below the plane $z=x-2y$ and above the triangle with vertices $(0,0), (1,1), (0,1)$ in the $xy-$plane. Need to divide into two volumes in one direction leading to the below theorem.
\item Theorem: If $D = D_1 \cup D_2$, then
\[
\iint\limits_D f(x,y) ~dA = \iint\limits_{D_1} f(x,y) ~dA + \iint\limits_{D_2} f(x,y) ~dA 
\]
\end{enumerate}

%%%%%%%%%%%%%%%%%%%%
\item If integration is hard, estimation will often do by capturing the solid inside and outside a cylinder.
\begin{enumerate}
\item Theorem: For $m \leq f(x,y) \leq M$ on domain $D$ with area $A(D)$, then
\[
m A(D) \leq \iint\limits_{D} f(x,y) ~dA \leq M A(D).
\]
\end{enumerate}

%%%%%%%%%%%%%%%%%%%%
\item

%%%%%%%%%%%%%%%%%%%%
\item

%%%%%%%%%%%%%%%%%%%%
\item Homework: 

\end{enumerate}


\subsection{15.3 Double integrals over polar coordinates}
\subsection{15.4 Applications of double integrals}
\subsection{15.5 Surface area}
\subsection{15.6 Triple integrals}
\subsection{15.7 Triple integrals in cylindrical coordinates}
\subsection{15.8 Triple integrals in spherical coordinates}
\subsection{15.9 Change of variable in multiple integrals}

%%%%%%%%%%%%%%%%%%%%%%%%%%%%%%%%%%%%%%%
%%%%%%%%%%%%%%%%%%%%%%%%%%%%%%%%%%%%%%%
\section{Chapter 16 Vector calculus}
\subsection{16.1 Vector fields}
\subsection{16.2 Line integrals}
\subsection{16.3 The fundamental theorem of line integrals}
\subsection{16.4 Green's theorem}
\subsection{16.5 Curl and divergence}
\subsection{16.6 Parametric surfaces and their area}
\subsection{16.7 Surface integrals}
\subsection{16.8 Stoke's theorem}
\subsection{16.9 The divergence theorem}
\subsection{16.10 Summary}

%%%%%%%%%%%%%%%%%%%%%%%%%%%%%%%%%%%%%%%
%%%%%%%%%%%%%%%%%%%%%%%%%%%%%%%%%%%%%%%
\section{Chapter 17 Second-order differential equations}
\subsection{17.1 Second-order linear equations}
\subsection{17.2 Nonhomogeneous linear equations}
\subsection{17.3 Applications of second-order differential equations}
\subsection{17.4 Series solutions}

\end{document}